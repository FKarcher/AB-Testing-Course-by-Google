\documentclass[11pt]{article}

    \usepackage[breakable]{tcolorbox}
    \usepackage{parskip} % Stop auto-indenting (to mimic markdown behaviour)
    
    \usepackage{iftex}
    \ifPDFTeX
    	\usepackage[T1]{fontenc}
    	\usepackage{mathpazo}
    \else
    	\usepackage{fontspec}
    \fi

    % Basic figure setup, for now with no caption control since it's done
    % automatically by Pandoc (which extracts ![](path) syntax from Markdown).
    \usepackage{graphicx}
    % Maintain compatibility with old templates. Remove in nbconvert 6.0
    \let\Oldincludegraphics\includegraphics
    % Ensure that by default, figures have no caption (until we provide a
    % proper Figure object with a Caption API and a way to capture that
    % in the conversion process - todo).
    \usepackage{caption}
    \DeclareCaptionFormat{nocaption}{}
    \captionsetup{format=nocaption,aboveskip=0pt,belowskip=0pt}

    \usepackage[Export]{adjustbox} % Used to constrain images to a maximum size
    \adjustboxset{max size={0.9\linewidth}{0.9\paperheight}}
    \usepackage{float}
    \floatplacement{figure}{H} % forces figures to be placed at the correct location
    \usepackage{xcolor} % Allow colors to be defined
    \usepackage{enumerate} % Needed for markdown enumerations to work
    \usepackage{geometry} % Used to adjust the document margins
    \usepackage{amsmath} % Equations
    \usepackage{amssymb} % Equations
    \usepackage{textcomp} % defines textquotesingle
    % Hack from http://tex.stackexchange.com/a/47451/13684:
    \AtBeginDocument{%
        \def\PYZsq{\textquotesingle}% Upright quotes in Pygmentized code
    }
    \usepackage{upquote} % Upright quotes for verbatim code
    \usepackage{eurosym} % defines \euro
    \usepackage[mathletters]{ucs} % Extended unicode (utf-8) support
    \usepackage{fancyvrb} % verbatim replacement that allows latex
    \usepackage{grffile} % extends the file name processing of package graphics 
                         % to support a larger range
    \makeatletter % fix for grffile with XeLaTeX
    \def\Gread@@xetex#1{%
      \IfFileExists{"\Gin@base".bb}%
      {\Gread@eps{\Gin@base.bb}}%
      {\Gread@@xetex@aux#1}%
    }
    \makeatother

    % The hyperref package gives us a pdf with properly built
    % internal navigation ('pdf bookmarks' for the table of contents,
    % internal cross-reference links, web links for URLs, etc.)
    \usepackage{hyperref}
    % The default LaTeX title has an obnoxious amount of whitespace. By default,
    % titling removes some of it. It also provides customization options.
    \usepackage{titling}
    \usepackage{longtable} % longtable support required by pandoc >1.10
    \usepackage{booktabs}  % table support for pandoc > 1.12.2
    \usepackage[inline]{enumitem} % IRkernel/repr support (it uses the enumerate* environment)
    \usepackage[normalem]{ulem} % ulem is needed to support strikethroughs (\sout)
                                % normalem makes italics be italics, not underlines
    \usepackage{mathrsfs}
    

    
    % Colors for the hyperref package
    \definecolor{urlcolor}{rgb}{0,.145,.698}
    \definecolor{linkcolor}{rgb}{.71,0.21,0.01}
    \definecolor{citecolor}{rgb}{.12,.54,.11}

    % ANSI colors
    \definecolor{ansi-black}{HTML}{3E424D}
    \definecolor{ansi-black-intense}{HTML}{282C36}
    \definecolor{ansi-red}{HTML}{E75C58}
    \definecolor{ansi-red-intense}{HTML}{B22B31}
    \definecolor{ansi-green}{HTML}{00A250}
    \definecolor{ansi-green-intense}{HTML}{007427}
    \definecolor{ansi-yellow}{HTML}{DDB62B}
    \definecolor{ansi-yellow-intense}{HTML}{B27D12}
    \definecolor{ansi-blue}{HTML}{208FFB}
    \definecolor{ansi-blue-intense}{HTML}{0065CA}
    \definecolor{ansi-magenta}{HTML}{D160C4}
    \definecolor{ansi-magenta-intense}{HTML}{A03196}
    \definecolor{ansi-cyan}{HTML}{60C6C8}
    \definecolor{ansi-cyan-intense}{HTML}{258F8F}
    \definecolor{ansi-white}{HTML}{C5C1B4}
    \definecolor{ansi-white-intense}{HTML}{A1A6B2}
    \definecolor{ansi-default-inverse-fg}{HTML}{FFFFFF}
    \definecolor{ansi-default-inverse-bg}{HTML}{000000}

    % commands and environments needed by pandoc snippets
    % extracted from the output of `pandoc -s`
    \providecommand{\tightlist}{%
      \setlength{\itemsep}{0pt}\setlength{\parskip}{0pt}}
    \DefineVerbatimEnvironment{Highlighting}{Verbatim}{commandchars=\\\{\}}
    % Add ',fontsize=\small' for more characters per line
    \newenvironment{Shaded}{}{}
    \newcommand{\KeywordTok}[1]{\textcolor[rgb]{0.00,0.44,0.13}{\textbf{{#1}}}}
    \newcommand{\DataTypeTok}[1]{\textcolor[rgb]{0.56,0.13,0.00}{{#1}}}
    \newcommand{\DecValTok}[1]{\textcolor[rgb]{0.25,0.63,0.44}{{#1}}}
    \newcommand{\BaseNTok}[1]{\textcolor[rgb]{0.25,0.63,0.44}{{#1}}}
    \newcommand{\FloatTok}[1]{\textcolor[rgb]{0.25,0.63,0.44}{{#1}}}
    \newcommand{\CharTok}[1]{\textcolor[rgb]{0.25,0.44,0.63}{{#1}}}
    \newcommand{\StringTok}[1]{\textcolor[rgb]{0.25,0.44,0.63}{{#1}}}
    \newcommand{\CommentTok}[1]{\textcolor[rgb]{0.38,0.63,0.69}{\textit{{#1}}}}
    \newcommand{\OtherTok}[1]{\textcolor[rgb]{0.00,0.44,0.13}{{#1}}}
    \newcommand{\AlertTok}[1]{\textcolor[rgb]{1.00,0.00,0.00}{\textbf{{#1}}}}
    \newcommand{\FunctionTok}[1]{\textcolor[rgb]{0.02,0.16,0.49}{{#1}}}
    \newcommand{\RegionMarkerTok}[1]{{#1}}
    \newcommand{\ErrorTok}[1]{\textcolor[rgb]{1.00,0.00,0.00}{\textbf{{#1}}}}
    \newcommand{\NormalTok}[1]{{#1}}
    
    % Additional commands for more recent versions of Pandoc
    \newcommand{\ConstantTok}[1]{\textcolor[rgb]{0.53,0.00,0.00}{{#1}}}
    \newcommand{\SpecialCharTok}[1]{\textcolor[rgb]{0.25,0.44,0.63}{{#1}}}
    \newcommand{\VerbatimStringTok}[1]{\textcolor[rgb]{0.25,0.44,0.63}{{#1}}}
    \newcommand{\SpecialStringTok}[1]{\textcolor[rgb]{0.73,0.40,0.53}{{#1}}}
    \newcommand{\ImportTok}[1]{{#1}}
    \newcommand{\DocumentationTok}[1]{\textcolor[rgb]{0.73,0.13,0.13}{\textit{{#1}}}}
    \newcommand{\AnnotationTok}[1]{\textcolor[rgb]{0.38,0.63,0.69}{\textbf{\textit{{#1}}}}}
    \newcommand{\CommentVarTok}[1]{\textcolor[rgb]{0.38,0.63,0.69}{\textbf{\textit{{#1}}}}}
    \newcommand{\VariableTok}[1]{\textcolor[rgb]{0.10,0.09,0.49}{{#1}}}
    \newcommand{\ControlFlowTok}[1]{\textcolor[rgb]{0.00,0.44,0.13}{\textbf{{#1}}}}
    \newcommand{\OperatorTok}[1]{\textcolor[rgb]{0.40,0.40,0.40}{{#1}}}
    \newcommand{\BuiltInTok}[1]{{#1}}
    \newcommand{\ExtensionTok}[1]{{#1}}
    \newcommand{\PreprocessorTok}[1]{\textcolor[rgb]{0.74,0.48,0.00}{{#1}}}
    \newcommand{\AttributeTok}[1]{\textcolor[rgb]{0.49,0.56,0.16}{{#1}}}
    \newcommand{\InformationTok}[1]{\textcolor[rgb]{0.38,0.63,0.69}{\textbf{\textit{{#1}}}}}
    \newcommand{\WarningTok}[1]{\textcolor[rgb]{0.38,0.63,0.69}{\textbf{\textit{{#1}}}}}
    
    
    % Define a nice break command that doesn't care if a line doesn't already
    % exist.
    \def\br{\hspace*{\fill} \\* }
    % Math Jax compatibility definitions
    \def\gt{>}
    \def\lt{<}
    \let\Oldtex\TeX
    \let\Oldlatex\LaTeX
    \renewcommand{\TeX}{\textrm{\Oldtex}}
    \renewcommand{\LaTeX}{\textrm{\Oldlatex}}
    % Document parameters
    % Document title
    \title{AB Testing using Python}
    
    
    
    
    
% Pygments definitions
\makeatletter
\def\PY@reset{\let\PY@it=\relax \let\PY@bf=\relax%
    \let\PY@ul=\relax \let\PY@tc=\relax%
    \let\PY@bc=\relax \let\PY@ff=\relax}
\def\PY@tok#1{\csname PY@tok@#1\endcsname}
\def\PY@toks#1+{\ifx\relax#1\empty\else%
    \PY@tok{#1}\expandafter\PY@toks\fi}
\def\PY@do#1{\PY@bc{\PY@tc{\PY@ul{%
    \PY@it{\PY@bf{\PY@ff{#1}}}}}}}
\def\PY#1#2{\PY@reset\PY@toks#1+\relax+\PY@do{#2}}

\expandafter\def\csname PY@tok@w\endcsname{\def\PY@tc##1{\textcolor[rgb]{0.73,0.73,0.73}{##1}}}
\expandafter\def\csname PY@tok@c\endcsname{\let\PY@it=\textit\def\PY@tc##1{\textcolor[rgb]{0.25,0.50,0.50}{##1}}}
\expandafter\def\csname PY@tok@cp\endcsname{\def\PY@tc##1{\textcolor[rgb]{0.74,0.48,0.00}{##1}}}
\expandafter\def\csname PY@tok@k\endcsname{\let\PY@bf=\textbf\def\PY@tc##1{\textcolor[rgb]{0.00,0.50,0.00}{##1}}}
\expandafter\def\csname PY@tok@kp\endcsname{\def\PY@tc##1{\textcolor[rgb]{0.00,0.50,0.00}{##1}}}
\expandafter\def\csname PY@tok@kt\endcsname{\def\PY@tc##1{\textcolor[rgb]{0.69,0.00,0.25}{##1}}}
\expandafter\def\csname PY@tok@o\endcsname{\def\PY@tc##1{\textcolor[rgb]{0.40,0.40,0.40}{##1}}}
\expandafter\def\csname PY@tok@ow\endcsname{\let\PY@bf=\textbf\def\PY@tc##1{\textcolor[rgb]{0.67,0.13,1.00}{##1}}}
\expandafter\def\csname PY@tok@nb\endcsname{\def\PY@tc##1{\textcolor[rgb]{0.00,0.50,0.00}{##1}}}
\expandafter\def\csname PY@tok@nf\endcsname{\def\PY@tc##1{\textcolor[rgb]{0.00,0.00,1.00}{##1}}}
\expandafter\def\csname PY@tok@nc\endcsname{\let\PY@bf=\textbf\def\PY@tc##1{\textcolor[rgb]{0.00,0.00,1.00}{##1}}}
\expandafter\def\csname PY@tok@nn\endcsname{\let\PY@bf=\textbf\def\PY@tc##1{\textcolor[rgb]{0.00,0.00,1.00}{##1}}}
\expandafter\def\csname PY@tok@ne\endcsname{\let\PY@bf=\textbf\def\PY@tc##1{\textcolor[rgb]{0.82,0.25,0.23}{##1}}}
\expandafter\def\csname PY@tok@nv\endcsname{\def\PY@tc##1{\textcolor[rgb]{0.10,0.09,0.49}{##1}}}
\expandafter\def\csname PY@tok@no\endcsname{\def\PY@tc##1{\textcolor[rgb]{0.53,0.00,0.00}{##1}}}
\expandafter\def\csname PY@tok@nl\endcsname{\def\PY@tc##1{\textcolor[rgb]{0.63,0.63,0.00}{##1}}}
\expandafter\def\csname PY@tok@ni\endcsname{\let\PY@bf=\textbf\def\PY@tc##1{\textcolor[rgb]{0.60,0.60,0.60}{##1}}}
\expandafter\def\csname PY@tok@na\endcsname{\def\PY@tc##1{\textcolor[rgb]{0.49,0.56,0.16}{##1}}}
\expandafter\def\csname PY@tok@nt\endcsname{\let\PY@bf=\textbf\def\PY@tc##1{\textcolor[rgb]{0.00,0.50,0.00}{##1}}}
\expandafter\def\csname PY@tok@nd\endcsname{\def\PY@tc##1{\textcolor[rgb]{0.67,0.13,1.00}{##1}}}
\expandafter\def\csname PY@tok@s\endcsname{\def\PY@tc##1{\textcolor[rgb]{0.73,0.13,0.13}{##1}}}
\expandafter\def\csname PY@tok@sd\endcsname{\let\PY@it=\textit\def\PY@tc##1{\textcolor[rgb]{0.73,0.13,0.13}{##1}}}
\expandafter\def\csname PY@tok@si\endcsname{\let\PY@bf=\textbf\def\PY@tc##1{\textcolor[rgb]{0.73,0.40,0.53}{##1}}}
\expandafter\def\csname PY@tok@se\endcsname{\let\PY@bf=\textbf\def\PY@tc##1{\textcolor[rgb]{0.73,0.40,0.13}{##1}}}
\expandafter\def\csname PY@tok@sr\endcsname{\def\PY@tc##1{\textcolor[rgb]{0.73,0.40,0.53}{##1}}}
\expandafter\def\csname PY@tok@ss\endcsname{\def\PY@tc##1{\textcolor[rgb]{0.10,0.09,0.49}{##1}}}
\expandafter\def\csname PY@tok@sx\endcsname{\def\PY@tc##1{\textcolor[rgb]{0.00,0.50,0.00}{##1}}}
\expandafter\def\csname PY@tok@m\endcsname{\def\PY@tc##1{\textcolor[rgb]{0.40,0.40,0.40}{##1}}}
\expandafter\def\csname PY@tok@gh\endcsname{\let\PY@bf=\textbf\def\PY@tc##1{\textcolor[rgb]{0.00,0.00,0.50}{##1}}}
\expandafter\def\csname PY@tok@gu\endcsname{\let\PY@bf=\textbf\def\PY@tc##1{\textcolor[rgb]{0.50,0.00,0.50}{##1}}}
\expandafter\def\csname PY@tok@gd\endcsname{\def\PY@tc##1{\textcolor[rgb]{0.63,0.00,0.00}{##1}}}
\expandafter\def\csname PY@tok@gi\endcsname{\def\PY@tc##1{\textcolor[rgb]{0.00,0.63,0.00}{##1}}}
\expandafter\def\csname PY@tok@gr\endcsname{\def\PY@tc##1{\textcolor[rgb]{1.00,0.00,0.00}{##1}}}
\expandafter\def\csname PY@tok@ge\endcsname{\let\PY@it=\textit}
\expandafter\def\csname PY@tok@gs\endcsname{\let\PY@bf=\textbf}
\expandafter\def\csname PY@tok@gp\endcsname{\let\PY@bf=\textbf\def\PY@tc##1{\textcolor[rgb]{0.00,0.00,0.50}{##1}}}
\expandafter\def\csname PY@tok@go\endcsname{\def\PY@tc##1{\textcolor[rgb]{0.53,0.53,0.53}{##1}}}
\expandafter\def\csname PY@tok@gt\endcsname{\def\PY@tc##1{\textcolor[rgb]{0.00,0.27,0.87}{##1}}}
\expandafter\def\csname PY@tok@err\endcsname{\def\PY@bc##1{\setlength{\fboxsep}{0pt}\fcolorbox[rgb]{1.00,0.00,0.00}{1,1,1}{\strut ##1}}}
\expandafter\def\csname PY@tok@kc\endcsname{\let\PY@bf=\textbf\def\PY@tc##1{\textcolor[rgb]{0.00,0.50,0.00}{##1}}}
\expandafter\def\csname PY@tok@kd\endcsname{\let\PY@bf=\textbf\def\PY@tc##1{\textcolor[rgb]{0.00,0.50,0.00}{##1}}}
\expandafter\def\csname PY@tok@kn\endcsname{\let\PY@bf=\textbf\def\PY@tc##1{\textcolor[rgb]{0.00,0.50,0.00}{##1}}}
\expandafter\def\csname PY@tok@kr\endcsname{\let\PY@bf=\textbf\def\PY@tc##1{\textcolor[rgb]{0.00,0.50,0.00}{##1}}}
\expandafter\def\csname PY@tok@bp\endcsname{\def\PY@tc##1{\textcolor[rgb]{0.00,0.50,0.00}{##1}}}
\expandafter\def\csname PY@tok@fm\endcsname{\def\PY@tc##1{\textcolor[rgb]{0.00,0.00,1.00}{##1}}}
\expandafter\def\csname PY@tok@vc\endcsname{\def\PY@tc##1{\textcolor[rgb]{0.10,0.09,0.49}{##1}}}
\expandafter\def\csname PY@tok@vg\endcsname{\def\PY@tc##1{\textcolor[rgb]{0.10,0.09,0.49}{##1}}}
\expandafter\def\csname PY@tok@vi\endcsname{\def\PY@tc##1{\textcolor[rgb]{0.10,0.09,0.49}{##1}}}
\expandafter\def\csname PY@tok@vm\endcsname{\def\PY@tc##1{\textcolor[rgb]{0.10,0.09,0.49}{##1}}}
\expandafter\def\csname PY@tok@sa\endcsname{\def\PY@tc##1{\textcolor[rgb]{0.73,0.13,0.13}{##1}}}
\expandafter\def\csname PY@tok@sb\endcsname{\def\PY@tc##1{\textcolor[rgb]{0.73,0.13,0.13}{##1}}}
\expandafter\def\csname PY@tok@sc\endcsname{\def\PY@tc##1{\textcolor[rgb]{0.73,0.13,0.13}{##1}}}
\expandafter\def\csname PY@tok@dl\endcsname{\def\PY@tc##1{\textcolor[rgb]{0.73,0.13,0.13}{##1}}}
\expandafter\def\csname PY@tok@s2\endcsname{\def\PY@tc##1{\textcolor[rgb]{0.73,0.13,0.13}{##1}}}
\expandafter\def\csname PY@tok@sh\endcsname{\def\PY@tc##1{\textcolor[rgb]{0.73,0.13,0.13}{##1}}}
\expandafter\def\csname PY@tok@s1\endcsname{\def\PY@tc##1{\textcolor[rgb]{0.73,0.13,0.13}{##1}}}
\expandafter\def\csname PY@tok@mb\endcsname{\def\PY@tc##1{\textcolor[rgb]{0.40,0.40,0.40}{##1}}}
\expandafter\def\csname PY@tok@mf\endcsname{\def\PY@tc##1{\textcolor[rgb]{0.40,0.40,0.40}{##1}}}
\expandafter\def\csname PY@tok@mh\endcsname{\def\PY@tc##1{\textcolor[rgb]{0.40,0.40,0.40}{##1}}}
\expandafter\def\csname PY@tok@mi\endcsname{\def\PY@tc##1{\textcolor[rgb]{0.40,0.40,0.40}{##1}}}
\expandafter\def\csname PY@tok@il\endcsname{\def\PY@tc##1{\textcolor[rgb]{0.40,0.40,0.40}{##1}}}
\expandafter\def\csname PY@tok@mo\endcsname{\def\PY@tc##1{\textcolor[rgb]{0.40,0.40,0.40}{##1}}}
\expandafter\def\csname PY@tok@ch\endcsname{\let\PY@it=\textit\def\PY@tc##1{\textcolor[rgb]{0.25,0.50,0.50}{##1}}}
\expandafter\def\csname PY@tok@cm\endcsname{\let\PY@it=\textit\def\PY@tc##1{\textcolor[rgb]{0.25,0.50,0.50}{##1}}}
\expandafter\def\csname PY@tok@cpf\endcsname{\let\PY@it=\textit\def\PY@tc##1{\textcolor[rgb]{0.25,0.50,0.50}{##1}}}
\expandafter\def\csname PY@tok@c1\endcsname{\let\PY@it=\textit\def\PY@tc##1{\textcolor[rgb]{0.25,0.50,0.50}{##1}}}
\expandafter\def\csname PY@tok@cs\endcsname{\let\PY@it=\textit\def\PY@tc##1{\textcolor[rgb]{0.25,0.50,0.50}{##1}}}

\def\PYZbs{\char`\\}
\def\PYZus{\char`\_}
\def\PYZob{\char`\{}
\def\PYZcb{\char`\}}
\def\PYZca{\char`\^}
\def\PYZam{\char`\&}
\def\PYZlt{\char`\<}
\def\PYZgt{\char`\>}
\def\PYZsh{\char`\#}
\def\PYZpc{\char`\%}
\def\PYZdl{\char`\$}
\def\PYZhy{\char`\-}
\def\PYZsq{\char`\'}
\def\PYZdq{\char`\"}
\def\PYZti{\char`\~}
% for compatibility with earlier versions
\def\PYZat{@}
\def\PYZlb{[}
\def\PYZrb{]}
\makeatother


    % For linebreaks inside Verbatim environment from package fancyvrb. 
    \makeatletter
        \newbox\Wrappedcontinuationbox 
        \newbox\Wrappedvisiblespacebox 
        \newcommand*\Wrappedvisiblespace {\textcolor{red}{\textvisiblespace}} 
        \newcommand*\Wrappedcontinuationsymbol {\textcolor{red}{\llap{\tiny$\m@th\hookrightarrow$}}} 
        \newcommand*\Wrappedcontinuationindent {3ex } 
        \newcommand*\Wrappedafterbreak {\kern\Wrappedcontinuationindent\copy\Wrappedcontinuationbox} 
        % Take advantage of the already applied Pygments mark-up to insert 
        % potential linebreaks for TeX processing. 
        %        {, <, #, %, $, ' and ": go to next line. 
        %        _, }, ^, &, >, - and ~: stay at end of broken line. 
        % Use of \textquotesingle for straight quote. 
        \newcommand*\Wrappedbreaksatspecials {% 
            \def\PYGZus{\discretionary{\char`\_}{\Wrappedafterbreak}{\char`\_}}% 
            \def\PYGZob{\discretionary{}{\Wrappedafterbreak\char`\{}{\char`\{}}% 
            \def\PYGZcb{\discretionary{\char`\}}{\Wrappedafterbreak}{\char`\}}}% 
            \def\PYGZca{\discretionary{\char`\^}{\Wrappedafterbreak}{\char`\^}}% 
            \def\PYGZam{\discretionary{\char`\&}{\Wrappedafterbreak}{\char`\&}}% 
            \def\PYGZlt{\discretionary{}{\Wrappedafterbreak\char`\<}{\char`\<}}% 
            \def\PYGZgt{\discretionary{\char`\>}{\Wrappedafterbreak}{\char`\>}}% 
            \def\PYGZsh{\discretionary{}{\Wrappedafterbreak\char`\#}{\char`\#}}% 
            \def\PYGZpc{\discretionary{}{\Wrappedafterbreak\char`\%}{\char`\%}}% 
            \def\PYGZdl{\discretionary{}{\Wrappedafterbreak\char`\$}{\char`\$}}% 
            \def\PYGZhy{\discretionary{\char`\-}{\Wrappedafterbreak}{\char`\-}}% 
            \def\PYGZsq{\discretionary{}{\Wrappedafterbreak\textquotesingle}{\textquotesingle}}% 
            \def\PYGZdq{\discretionary{}{\Wrappedafterbreak\char`\"}{\char`\"}}% 
            \def\PYGZti{\discretionary{\char`\~}{\Wrappedafterbreak}{\char`\~}}% 
        } 
        % Some characters . , ; ? ! / are not pygmentized. 
        % This macro makes them "active" and they will insert potential linebreaks 
        \newcommand*\Wrappedbreaksatpunct {% 
            \lccode`\~`\.\lowercase{\def~}{\discretionary{\hbox{\char`\.}}{\Wrappedafterbreak}{\hbox{\char`\.}}}% 
            \lccode`\~`\,\lowercase{\def~}{\discretionary{\hbox{\char`\,}}{\Wrappedafterbreak}{\hbox{\char`\,}}}% 
            \lccode`\~`\;\lowercase{\def~}{\discretionary{\hbox{\char`\;}}{\Wrappedafterbreak}{\hbox{\char`\;}}}% 
            \lccode`\~`\:\lowercase{\def~}{\discretionary{\hbox{\char`\:}}{\Wrappedafterbreak}{\hbox{\char`\:}}}% 
            \lccode`\~`\?\lowercase{\def~}{\discretionary{\hbox{\char`\?}}{\Wrappedafterbreak}{\hbox{\char`\?}}}% 
            \lccode`\~`\!\lowercase{\def~}{\discretionary{\hbox{\char`\!}}{\Wrappedafterbreak}{\hbox{\char`\!}}}% 
            \lccode`\~`\/\lowercase{\def~}{\discretionary{\hbox{\char`\/}}{\Wrappedafterbreak}{\hbox{\char`\/}}}% 
            \catcode`\.\active
            \catcode`\,\active 
            \catcode`\;\active
            \catcode`\:\active
            \catcode`\?\active
            \catcode`\!\active
            \catcode`\/\active 
            \lccode`\~`\~ 	
        }
    \makeatother

    \let\OriginalVerbatim=\Verbatim
    \makeatletter
    \renewcommand{\Verbatim}[1][1]{%
        %\parskip\z@skip
        \sbox\Wrappedcontinuationbox {\Wrappedcontinuationsymbol}%
        \sbox\Wrappedvisiblespacebox {\FV@SetupFont\Wrappedvisiblespace}%
        \def\FancyVerbFormatLine ##1{\hsize\linewidth
            \vtop{\raggedright\hyphenpenalty\z@\exhyphenpenalty\z@
                \doublehyphendemerits\z@\finalhyphendemerits\z@
                \strut ##1\strut}%
        }%
        % If the linebreak is at a space, the latter will be displayed as visible
        % space at end of first line, and a continuation symbol starts next line.
        % Stretch/shrink are however usually zero for typewriter font.
        \def\FV@Space {%
            \nobreak\hskip\z@ plus\fontdimen3\font minus\fontdimen4\font
            \discretionary{\copy\Wrappedvisiblespacebox}{\Wrappedafterbreak}
            {\kern\fontdimen2\font}%
        }%
        
        % Allow breaks at special characters using \PYG... macros.
        \Wrappedbreaksatspecials
        % Breaks at punctuation characters . , ; ? ! and / need catcode=\active 	
        \OriginalVerbatim[#1,codes*=\Wrappedbreaksatpunct]%
    }
    \makeatother

    % Exact colors from NB
    \definecolor{incolor}{HTML}{303F9F}
    \definecolor{outcolor}{HTML}{D84315}
    \definecolor{cellborder}{HTML}{CFCFCF}
    \definecolor{cellbackground}{HTML}{F7F7F7}
    
    % prompt
    \makeatletter
    \newcommand{\boxspacing}{\kern\kvtcb@left@rule\kern\kvtcb@boxsep}
    \makeatother
    \newcommand{\prompt}[4]{
        \ttfamily\llap{{\color{#2}[#3]:\hspace{3pt}#4}}\vspace{-\baselineskip}
    }
    

    
    % Prevent overflowing lines due to hard-to-break entities
    \sloppy 
    % Setup hyperref package
    \hypersetup{
      breaklinks=true,  % so long urls are correctly broken across lines
      colorlinks=true,
      urlcolor=urlcolor,
      linkcolor=linkcolor,
      citecolor=citecolor,
      }
    % Slightly bigger margins than the latex defaults
    
    \geometry{verbose,tmargin=1in,bmargin=1in,lmargin=1in,rmargin=1in}
    
    

\begin{document}
    
    \maketitle
    
    

    
    \hypertarget{capstone-project-ab-testing-course-by-google}{%
\section{Capstone Project: A/B Testing Course by
Google}\label{capstone-project-ab-testing-course-by-google}}

\begin{enumerate}
\def\labelenumi{\arabic{enumi}.}
\tightlist
\item
  Section \ref{1}
\item
  Section \ref{2} Section \ref{21} Section \ref{22} Section \ref{23}
  Section \ref{24}
\item
  Section \ref{3} Section \ref{31} Section \ref{32}
\item
  Section \ref{4} Section \ref{41} Section \ref{42}
\item
  Section \ref{5} Section \ref{51} Section \ref{52}
\item
  Section \ref{6} Section \ref{61} Section \ref{62} Section \ref{63}
\item
  Section \ref{7}
\item
  Section \ref{8}
\end{enumerate}

    \begin{tcolorbox}[breakable, size=fbox, boxrule=1pt, pad at break*=1mm,colback=cellbackground, colframe=cellborder]
\prompt{In}{incolor}{1}{\boxspacing}
\begin{Verbatim}[commandchars=\\\{\}]
\PY{k+kn}{import} \PY{n+nn}{pandas} \PY{k}{as} \PY{n+nn}{pd}
\PY{k+kn}{import} \PY{n+nn}{numpy} \PY{k}{as} \PY{n+nn}{np}
\PY{k+kn}{import} \PY{n+nn}{math}
\end{Verbatim}
\end{tcolorbox}

    \hypertarget{udacitys-ab-testing-course-by-google}{%
\subsection{1. Udacity's A/B Testing Course by
Google}\label{udacitys-ab-testing-course-by-google}}

I recently finished the A/B Testing course by Google on Udacity. I
highly recommend this course to people who want to learn how A/B testing
is done, and to data scientists who are interested in how data science,
python and statistics are applied in real-life business scenarios. The
course summarized how to run an A/B test into 5 steps: 1. Choose
invariant and evalution metrics. 2. Choose significance level (alpha),
statistical power (1-beta) and practical significance level (the minimum
change we want to observe in order to launch the change). 3. Calculate
required sample size. 4. Run the test for control and experiment groups.
5. Analyze the results, calulate confidernce intervals of evaludation
metrics, and draw conclusions.

    This notebook is a capstone project for this course. The goal of this
capstone project is to utlize the skills learned in this course and
apply them to a business case using real life data. We want to gain
insights about if adding an extra screening step after a student clicks
the ``start free trial'' button will reduce the number of frustrated
students who left the free trial because they couldn't commit enough
hours to the course.

    \hypertarget{experiment-overview-free-trial-screener}{%
\subsection{2. Experiment Overview: Free Trial
Screener}\label{experiment-overview-free-trial-screener}}

    \hypertarget{description-of-original-conditions}{%
\subsubsection{2.1 Description of original
conditions}\label{description-of-original-conditions}}

\begin{itemize}
\tightlist
\item
  At the time of this experiment, Udacity currently has two options on
  the course overview page: ``start free trial'', and ``access course
  materials''.
\item
  If the student clicks ``start free trial'', they will be asked to
  enter their credit card information, and then they will be enrolled in
  a free trial for the paid version of the course. After 14 days, they
  will automatically be charged unless they cancel first.
\item
  If the student clicks ``access course materials'', they will be able
  to view the videos and take the quizzes for free, but they will not
  receive coaching support or a verified certificate, and they will not
  submit their final project for feedback.
\end{itemize}

    \hypertarget{description-of-the-experimental-change}{%
\subsubsection{2.2 Description of the experimental
change}\label{description-of-the-experimental-change}}

\begin{itemize}
\tightlist
\item
  In the experiment, Udacity tested a change where if the student
  clicked ``start free trial'', they were asked how much time they had
  available to devote to the course.
\item
  If the student indicated 5 or more hours per week, they would be taken
  through the checkout process as usual.
\item
  If they indicated fewer than 5 hours per week, a message would appear
  indicating that Udacity courses usually require a greater time
  commitment for successful completion, and suggesting that the student
  might like to access the course materials for free.
\item
  At this point, the student would have the option to continue enrolling
  in the free trial, or access the course materials for free instead.
  This screenshot shows what the experiment looks like.
  \includegraphics{Experiment.png}
\end{itemize}

    \hypertarget{experiment-hypothesis}{%
\subsubsection{2.3 Experiment hypothesis}\label{experiment-hypothesis}}

The hypothesis was that this might set clearer expectations for students
upfront, thus reducing the number of frustrated students who left the
free trial because they didn't have enough time---without significantly
reducing the number of students to continue past the free trial and
eventually pay and complete the course. If this hypothesis held true,
Udacity could improve the overall student experience and improve
coaches' capacity to support students who are likely to complete the
course.

    \hypertarget{how-data-are-tracked}{%
\subsubsection{2.4 How data are tracked}\label{how-data-are-tracked}}

\textbf{The unit of diversion is a cookie}, although if the student
enrolls in the free trial, they are tracked by user-id from that point
forward. The same user-id cannot enroll in the free trial twice. For
users that do not enroll, their user-id is not tracked in the
experiment, even if they were signed in when they visited the course
overview page.

    \hypertarget{metric-choice}{%
\subsection{3. Metric Choice}\label{metric-choice}}

    Which of the following metrics should we choose to measure for this
experiment and why? For each metric we choose, we need to decide whether
it is an invariant metric or an evaluation metric. The practical
significance boundary for each metric, that is, the difference that
would have to be observed before that was a meaningful change for the
business, is given in parentheses. All practical significance boundaries
are given as absolute changes. Any place ``unique cookies'' are
mentioned, the uniqueness is determined by day. (That is, the same
cookie visiting on different days would be counted twice.) User-ids are
automatically unique since the site does not allow the same user-id to
enroll twice.

    \begin{itemize}
\tightlist
\item
  Number of cookies: That is, number of unique cookies to view the
  course overview page. (dmin=3000)
\item
  Number of user-ids: That is, number of users who enroll in the free
  trial. (dmin=50)
\item
  Number of clicks: That is, number of unique cookies to click the
  ``Start free trial'' button (which happens before the free trial
  screener is trigger). (dmin=240)
\item
  Click-through-probability: That is, number of unique cookies to click
  the ``Start free trial'' button divided by number of unique cookies to
  view the course overview page. (dmin=0.01)
\item
  Gross conversion: That is, number of user-ids to complete checkout and
  enroll in the free trial divided by number of unique cookies to click
  the ``Start free trial'' button. (dmin= 0.01)
\item
  Retention: That is, number of user-ids to remain enrolled past the
  14-day boundary (and thus make at least one payment) divided by number
  of user-ids to complete checkout. (dmin=0.01)
\item
  Net conversion: That is, number of user-ids to remain enrolled past
  the 14-day boundary (and thus make at least one payment) divided by
  the number of unique cookies to click the ``Start free trial'' button.
  (dmin= 0.0075)
\end{itemize}

    \hypertarget{choose-invariant-metrics}{%
\subsubsection{3.1 Choose invariant
metrics}\label{choose-invariant-metrics}}

    Invariant metrics are used for ``sanity checks'', i.e., to make sure the
way we collect data for the experiment is randomized. We should pick
metrics that are not affected by the experiment, so they stay the same
in both our control and experiment groups.

Number of cookies, number of clicks and click-through-probability (CTR)
are chosen as invariant metrics, as they are tracked BEFORE students see
the experiment change; these metrics are not affected by the
experimental change.

Number of user-ids are not chosen as an invariant metric, because they
are only tracked AFTER users enroll in a free trial; if they are not
enrolled, user-ids are not tracked.

    \begin{longtable}[]{@{}cccc@{}}
\toprule
\begin{minipage}[b]{0.39\columnwidth}\centering
Invariant Metrics\strut
\end{minipage} & \begin{minipage}[b]{0.18\columnwidth}\centering
Notation\strut
\end{minipage} & \begin{minipage}[b]{0.23\columnwidth}\centering
Definition\strut
\end{minipage} & \begin{minipage}[b]{0.09\columnwidth}\centering
Dmin\strut
\end{minipage}\tabularnewline
\midrule
\endhead
\begin{minipage}[t]{0.39\columnwidth}\centering
\# of cookies\strut
\end{minipage} & \begin{minipage}[t]{0.18\columnwidth}\centering
Ck\strut
\end{minipage} & \begin{minipage}[t]{0.23\columnwidth}\centering
\# of unique cookies to view the course overview page\strut
\end{minipage} & \begin{minipage}[t]{0.09\columnwidth}\centering
3000\strut
\end{minipage}\tabularnewline
\begin{minipage}[t]{0.39\columnwidth}\centering
\# of clicks\strut
\end{minipage} & \begin{minipage}[t]{0.18\columnwidth}\centering
Cl\strut
\end{minipage} & \begin{minipage}[t]{0.23\columnwidth}\centering
\# of unique cookies to click the ``Start free trial'' button\strut
\end{minipage} & \begin{minipage}[t]{0.09\columnwidth}\centering
240\strut
\end{minipage}\tabularnewline
\begin{minipage}[t]{0.39\columnwidth}\centering
Click-through-probability\strut
\end{minipage} & \begin{minipage}[t]{0.18\columnwidth}\centering
CTP\strut
\end{minipage} & \begin{minipage}[t]{0.23\columnwidth}\centering
Cl/Ck\strut
\end{minipage} & \begin{minipage}[t]{0.09\columnwidth}\centering
0.01\strut
\end{minipage}\tabularnewline
\bottomrule
\end{longtable}

    \hypertarget{choose-evaluation-metrics}{%
\subsubsection{3.2 Choose evaluation
metrics}\label{choose-evaluation-metrics}}

    Evaluation metrics are the metrics we expect to change, and represent
practical business goals that we aim to achieve. We choose gross
conversion, retention and net conversion as evaluation metrics, as they
are tracked AFTER the customers click the ``Start free trial'' and there
is a change between the control group and the experiment group.

    \begin{longtable}[]{@{}cccc@{}}
\toprule
\begin{minipage}[b]{0.39\columnwidth}\centering
Evaluation Metrics\strut
\end{minipage} & \begin{minipage}[b]{0.18\columnwidth}\centering
Notation\strut
\end{minipage} & \begin{minipage}[b]{0.23\columnwidth}\centering
Definition\strut
\end{minipage} & \begin{minipage}[b]{0.09\columnwidth}\centering
Dmin\strut
\end{minipage}\tabularnewline
\midrule
\endhead
\begin{minipage}[t]{0.39\columnwidth}\centering
Gross conversion\strut
\end{minipage} & \begin{minipage}[t]{0.18\columnwidth}\centering
GrCon\strut
\end{minipage} & \begin{minipage}[t]{0.23\columnwidth}\centering
\# of enrolled/Cl\strut
\end{minipage} & \begin{minipage}[t]{0.09\columnwidth}\centering
0.01\strut
\end{minipage}\tabularnewline
\begin{minipage}[t]{0.39\columnwidth}\centering
Retention\strut
\end{minipage} & \begin{minipage}[t]{0.18\columnwidth}\centering
Ret\strut
\end{minipage} & \begin{minipage}[t]{0.23\columnwidth}\centering
\# of paid/\# of enrolled = NetCon/GrCon\strut
\end{minipage} & \begin{minipage}[t]{0.09\columnwidth}\centering
0.01\strut
\end{minipage}\tabularnewline
\begin{minipage}[t]{0.39\columnwidth}\centering
Net conversion\strut
\end{minipage} & \begin{minipage}[t]{0.18\columnwidth}\centering
NetCon\strut
\end{minipage} & \begin{minipage}[t]{0.23\columnwidth}\centering
\# of paid (and thus make at least one payment)/Cl\strut
\end{minipage} & \begin{minipage}[t]{0.09\columnwidth}\centering
0.0075\strut
\end{minipage}\tabularnewline
\bottomrule
\end{longtable}

    \hypertarget{measuring-standard-deviation}{%
\subsection{4. Measuring Standard
Deviation}\label{measuring-standard-deviation}}

    \hypertarget{baseline-values}{%
\subsubsection{4.1 Baseline values}\label{baseline-values}}

    Udacity provided rough estimates for these metrics (the numbers have
been changed from Udacity's true number); these are how the metrics
behave before the change, aka, our baseline values.

    \begin{longtable}[]{@{}ccc@{}}
\toprule
Metrics & Definition & Estimator\tabularnewline
\midrule
\endhead
\# of cookies & Unique cookies to view course overview page per day &
40000\tabularnewline
\# of clicks & Unique cookies to click ``Start free trial'' per day &
3200\tabularnewline
\# of enrollments & Enrollments in the free trial per day &
660\tabularnewline
CTP & Click-through-probability on ``Start free trial'' &
0.08\tabularnewline
Gross conversion & Probability of enrollment, given click &
0.20625\tabularnewline
Retention & Probability of payment, given enroll & 0.53\tabularnewline
Net conversion & Probability of payment, given click &
0.1093125\tabularnewline
\bottomrule
\end{longtable}

    \hypertarget{calculate-the-standard-deviation}{%
\subsubsection{4.2 Calculate the standard
deviation}\label{calculate-the-standard-deviation}}

    For metrics we selected as our evalution metric: gross conversion,
retention and net conversion, let's make an analytic estimate of its
standard deviation, given \textbf{a sample size of 5000 cookies}
visiting the course overview page.

    First, we need to scale estimators to our sample size. In this case,
from 40000 "unique cookies to view course overview page per day: to 5000
for our sample size.

    \begin{tcolorbox}[breakable, size=fbox, boxrule=1pt, pad at break*=1mm,colback=cellbackground, colframe=cellborder]
\prompt{In}{incolor}{2}{\boxspacing}
\begin{Verbatim}[commandchars=\\\{\}]
\PY{c+c1}{\PYZsh{} Put base estimators into a dictionary}
\PY{n}{baseline} \PY{o}{=} \PY{p}{\PYZob{}}\PY{l+s+s1}{\PYZsq{}}\PY{l+s+s1}{cookies}\PY{l+s+s1}{\PYZsq{}}\PY{p}{:}\PY{l+m+mi}{40000}\PY{p}{,} \PY{l+s+s1}{\PYZsq{}}\PY{l+s+s1}{clicks}\PY{l+s+s1}{\PYZsq{}}\PY{p}{:}\PY{l+m+mi}{3200}\PY{p}{,} \PY{l+s+s1}{\PYZsq{}}\PY{l+s+s1}{enrollments}\PY{l+s+s1}{\PYZsq{}}\PY{p}{:}\PY{l+m+mi}{660}\PY{p}{,} \PY{l+s+s1}{\PYZsq{}}\PY{l+s+s1}{CTP}\PY{l+s+s1}{\PYZsq{}}\PY{p}{:}\PY{l+m+mf}{0.08}\PY{p}{,} \PY{l+s+s2}{\PYZdq{}}\PY{l+s+s2}{GrCon}\PY{l+s+s2}{\PYZdq{}}\PY{p}{:}\PY{l+m+mf}{0.20625}\PY{p}{,} \PY{l+s+s2}{\PYZdq{}}\PY{l+s+s2}{Retention}\PY{l+s+s2}{\PYZdq{}}\PY{p}{:}\PY{l+m+mf}{0.53}\PY{p}{,} \PY{l+s+s2}{\PYZdq{}}\PY{l+s+s2}{NetCon}\PY{l+s+s2}{\PYZdq{}}\PY{p}{:}\PY{l+m+mf}{0.1093125}\PY{p}{\PYZcb{}}
\end{Verbatim}
\end{tcolorbox}

    \begin{tcolorbox}[breakable, size=fbox, boxrule=1pt, pad at break*=1mm,colback=cellbackground, colframe=cellborder]
\prompt{In}{incolor}{3}{\boxspacing}
\begin{Verbatim}[commandchars=\\\{\}]
\PY{c+c1}{\PYZsh{} Scale estimators to our sample size}
\PY{n}{baseline2} \PY{o}{=} \PY{n}{baseline}\PY{o}{.}\PY{n}{copy}\PY{p}{(}\PY{p}{)}
\PY{n}{baseline2}\PY{p}{[}\PY{l+s+s1}{\PYZsq{}}\PY{l+s+s1}{cookies}\PY{l+s+s1}{\PYZsq{}}\PY{p}{]} \PY{o}{=} \PY{l+m+mi}{5000}
\PY{n}{baseline2}\PY{p}{[}\PY{l+s+s1}{\PYZsq{}}\PY{l+s+s1}{clicks}\PY{l+s+s1}{\PYZsq{}}\PY{p}{]} \PY{o}{=} \PY{n}{baseline}\PY{p}{[}\PY{l+s+s1}{\PYZsq{}}\PY{l+s+s1}{clicks}\PY{l+s+s1}{\PYZsq{}}\PY{p}{]}\PY{o}{*}\PY{p}{(}\PY{l+m+mi}{5000}\PY{o}{/}\PY{l+m+mi}{40000}\PY{p}{)}
\PY{n}{baseline2}\PY{p}{[}\PY{l+s+s1}{\PYZsq{}}\PY{l+s+s1}{enrollments}\PY{l+s+s1}{\PYZsq{}}\PY{p}{]} \PY{o}{=} \PY{n}{baseline}\PY{p}{[}\PY{l+s+s1}{\PYZsq{}}\PY{l+s+s1}{enrollments}\PY{l+s+s1}{\PYZsq{}}\PY{p}{]}\PY{o}{*}\PY{p}{(}\PY{l+m+mi}{5000}\PY{o}{/}\PY{l+m+mi}{40000}\PY{p}{)}
\PY{n}{baseline2}
\end{Verbatim}
\end{tcolorbox}

            \begin{tcolorbox}[breakable, size=fbox, boxrule=.5pt, pad at break*=1mm, opacityfill=0]
\prompt{Out}{outcolor}{3}{\boxspacing}
\begin{Verbatim}[commandchars=\\\{\}]
\{'cookies': 5000,
 'clicks': 400.0,
 'enrollments': 82.5,
 'CTP': 0.08,
 'GrCon': 0.20625,
 'Retention': 0.53,
 'NetCon': 0.1093125\}
\end{Verbatim}
\end{tcolorbox}
        
    In order to estimate variances analytically, we can assume probability
(\(\hat {p}\)) are binomially distributed; so we can use the following
equation to calculate standard deviation. This equation only works when
the unit of diversion of the experiment is equal to unit of the analysis
(the denominator of the metric formula). In our case, both the unit of
diversion and the unit of analysis are cookies.

    \(SD=\sqrt{\frac{\hat{p}*(1-\hat{p})}{n}}\)

 \(\hat {p}\): baseline proability for the event to occur

\emph{n}: sample size

    \begin{itemize}
\tightlist
\item
  Gross Conversion
\end{itemize}

    \begin{tcolorbox}[breakable, size=fbox, boxrule=1pt, pad at break*=1mm,colback=cellbackground, colframe=cellborder]
\prompt{In}{incolor}{4}{\boxspacing}
\begin{Verbatim}[commandchars=\\\{\}]
\PY{n}{G} \PY{o}{=} \PY{p}{\PYZob{}}\PY{p}{\PYZcb{}}
\PY{n}{G}\PY{p}{[}\PY{l+s+s1}{\PYZsq{}}\PY{l+s+s1}{p}\PY{l+s+s1}{\PYZsq{}}\PY{p}{]} \PY{o}{=} \PY{n}{baseline2}\PY{p}{[}\PY{l+s+s1}{\PYZsq{}}\PY{l+s+s1}{GrCon}\PY{l+s+s1}{\PYZsq{}}\PY{p}{]}
\PY{n}{G}\PY{p}{[}\PY{l+s+s1}{\PYZsq{}}\PY{l+s+s1}{n}\PY{l+s+s1}{\PYZsq{}}\PY{p}{]} \PY{o}{=} \PY{n}{baseline2}\PY{p}{[}\PY{l+s+s1}{\PYZsq{}}\PY{l+s+s1}{clicks}\PY{l+s+s1}{\PYZsq{}}\PY{p}{]} \PY{c+c1}{\PYZsh{} the unit of diversion is clicks (after scaled)}
\PY{n}{G}\PY{p}{[}\PY{l+s+s1}{\PYZsq{}}\PY{l+s+s1}{sd}\PY{l+s+s1}{\PYZsq{}}\PY{p}{]} \PY{o}{=} \PY{n+nb}{round}\PY{p}{(}\PY{n}{math}\PY{o}{.}\PY{n}{sqrt}\PY{p}{(}\PY{n}{G}\PY{p}{[}\PY{l+s+s1}{\PYZsq{}}\PY{l+s+s1}{p}\PY{l+s+s1}{\PYZsq{}}\PY{p}{]}\PY{o}{*}\PY{p}{(}\PY{l+m+mi}{1}\PY{o}{\PYZhy{}}\PY{n}{G}\PY{p}{[}\PY{l+s+s1}{\PYZsq{}}\PY{l+s+s1}{p}\PY{l+s+s1}{\PYZsq{}}\PY{p}{]}\PY{p}{)}\PY{o}{/}\PY{n}{G}\PY{p}{[}\PY{l+s+s1}{\PYZsq{}}\PY{l+s+s1}{n}\PY{l+s+s1}{\PYZsq{}}\PY{p}{]}\PY{p}{)}\PY{p}{,}\PY{l+m+mi}{4}\PY{p}{)}
\PY{n}{G}\PY{p}{[}\PY{l+s+s1}{\PYZsq{}}\PY{l+s+s1}{sd}\PY{l+s+s1}{\PYZsq{}}\PY{p}{]}
\end{Verbatim}
\end{tcolorbox}

            \begin{tcolorbox}[breakable, size=fbox, boxrule=.5pt, pad at break*=1mm, opacityfill=0]
\prompt{Out}{outcolor}{4}{\boxspacing}
\begin{Verbatim}[commandchars=\\\{\}]
0.0202
\end{Verbatim}
\end{tcolorbox}
        
    \begin{itemize}
\tightlist
\item
  Retention
\end{itemize}

    \begin{tcolorbox}[breakable, size=fbox, boxrule=1pt, pad at break*=1mm,colback=cellbackground, colframe=cellborder]
\prompt{In}{incolor}{5}{\boxspacing}
\begin{Verbatim}[commandchars=\\\{\}]
\PY{n}{R} \PY{o}{=} \PY{p}{\PYZob{}}\PY{p}{\PYZcb{}}
\PY{n}{R}\PY{p}{[}\PY{l+s+s1}{\PYZsq{}}\PY{l+s+s1}{p}\PY{l+s+s1}{\PYZsq{}}\PY{p}{]} \PY{o}{=} \PY{n}{baseline2}\PY{p}{[}\PY{l+s+s1}{\PYZsq{}}\PY{l+s+s1}{Retention}\PY{l+s+s1}{\PYZsq{}}\PY{p}{]}
\PY{n}{R}\PY{p}{[}\PY{l+s+s1}{\PYZsq{}}\PY{l+s+s1}{n}\PY{l+s+s1}{\PYZsq{}}\PY{p}{]} \PY{o}{=} \PY{n}{baseline2}\PY{p}{[}\PY{l+s+s1}{\PYZsq{}}\PY{l+s+s1}{enrollments}\PY{l+s+s1}{\PYZsq{}}\PY{p}{]} \PY{c+c1}{\PYZsh{} the unit of diversion is enrollment (after scaled)}
\PY{n}{R}\PY{p}{[}\PY{l+s+s1}{\PYZsq{}}\PY{l+s+s1}{sd}\PY{l+s+s1}{\PYZsq{}}\PY{p}{]} \PY{o}{=} \PY{n+nb}{round}\PY{p}{(}\PY{n}{math}\PY{o}{.}\PY{n}{sqrt}\PY{p}{(}\PY{n}{R}\PY{p}{[}\PY{l+s+s1}{\PYZsq{}}\PY{l+s+s1}{p}\PY{l+s+s1}{\PYZsq{}}\PY{p}{]}\PY{o}{*}\PY{p}{(}\PY{l+m+mi}{1}\PY{o}{\PYZhy{}}\PY{n}{R}\PY{p}{[}\PY{l+s+s1}{\PYZsq{}}\PY{l+s+s1}{p}\PY{l+s+s1}{\PYZsq{}}\PY{p}{]}\PY{p}{)}\PY{o}{/}\PY{n}{R}\PY{p}{[}\PY{l+s+s1}{\PYZsq{}}\PY{l+s+s1}{n}\PY{l+s+s1}{\PYZsq{}}\PY{p}{]}\PY{p}{)}\PY{p}{,}\PY{l+m+mi}{4}\PY{p}{)}
\PY{n}{R}\PY{p}{[}\PY{l+s+s1}{\PYZsq{}}\PY{l+s+s1}{sd}\PY{l+s+s1}{\PYZsq{}}\PY{p}{]}
\end{Verbatim}
\end{tcolorbox}

            \begin{tcolorbox}[breakable, size=fbox, boxrule=.5pt, pad at break*=1mm, opacityfill=0]
\prompt{Out}{outcolor}{5}{\boxspacing}
\begin{Verbatim}[commandchars=\\\{\}]
0.0549
\end{Verbatim}
\end{tcolorbox}
        
    \begin{itemize}
\tightlist
\item
  Net Conversion
\end{itemize}

    \begin{tcolorbox}[breakable, size=fbox, boxrule=1pt, pad at break*=1mm,colback=cellbackground, colframe=cellborder]
\prompt{In}{incolor}{6}{\boxspacing}
\begin{Verbatim}[commandchars=\\\{\}]
\PY{n}{N} \PY{o}{=} \PY{p}{\PYZob{}}\PY{p}{\PYZcb{}}
\PY{n}{N}\PY{p}{[}\PY{l+s+s1}{\PYZsq{}}\PY{l+s+s1}{p}\PY{l+s+s1}{\PYZsq{}}\PY{p}{]} \PY{o}{=} \PY{n}{baseline2}\PY{p}{[}\PY{l+s+s1}{\PYZsq{}}\PY{l+s+s1}{NetCon}\PY{l+s+s1}{\PYZsq{}}\PY{p}{]}
\PY{n}{N}\PY{p}{[}\PY{l+s+s1}{\PYZsq{}}\PY{l+s+s1}{n}\PY{l+s+s1}{\PYZsq{}}\PY{p}{]} \PY{o}{=} \PY{n}{baseline2}\PY{p}{[}\PY{l+s+s1}{\PYZsq{}}\PY{l+s+s1}{clicks}\PY{l+s+s1}{\PYZsq{}}\PY{p}{]} \PY{c+c1}{\PYZsh{} the unit of diversion is clicks (after scaled)}
\PY{n}{N}\PY{p}{[}\PY{l+s+s1}{\PYZsq{}}\PY{l+s+s1}{sd}\PY{l+s+s1}{\PYZsq{}}\PY{p}{]} \PY{o}{=} \PY{n+nb}{round}\PY{p}{(}\PY{n}{math}\PY{o}{.}\PY{n}{sqrt}\PY{p}{(}\PY{n}{N}\PY{p}{[}\PY{l+s+s1}{\PYZsq{}}\PY{l+s+s1}{p}\PY{l+s+s1}{\PYZsq{}}\PY{p}{]}\PY{o}{*}\PY{p}{(}\PY{l+m+mi}{1}\PY{o}{\PYZhy{}}\PY{n}{N}\PY{p}{[}\PY{l+s+s1}{\PYZsq{}}\PY{l+s+s1}{p}\PY{l+s+s1}{\PYZsq{}}\PY{p}{]}\PY{p}{)}\PY{o}{/}\PY{n}{N}\PY{p}{[}\PY{l+s+s1}{\PYZsq{}}\PY{l+s+s1}{n}\PY{l+s+s1}{\PYZsq{}}\PY{p}{]}\PY{p}{)}\PY{p}{,}\PY{l+m+mi}{4}\PY{p}{)}
\PY{n}{N}\PY{p}{[}\PY{l+s+s1}{\PYZsq{}}\PY{l+s+s1}{sd}\PY{l+s+s1}{\PYZsq{}}\PY{p}{]}
\end{Verbatim}
\end{tcolorbox}

            \begin{tcolorbox}[breakable, size=fbox, boxrule=.5pt, pad at break*=1mm, opacityfill=0]
\prompt{Out}{outcolor}{6}{\boxspacing}
\begin{Verbatim}[commandchars=\\\{\}]
0.0156
\end{Verbatim}
\end{tcolorbox}
        
    \hypertarget{experiment-sizing}{%
\subsection{5. Experiment Sizing}\label{experiment-sizing}}

    \hypertarget{choose-sample-size}{%
\subsubsection{5.1 Choose sample size}\label{choose-sample-size}}

    In order to calculate the sample size, we need to know the significance
level alpha (set to 0.05), statistical power (1-beta, set to 0.8),
baseline conversion rate (provided by Udacity) and minimum detectable
effect (Dmin, provided by Udacity). We plug these 4 values into this
\href{http://www.evanmiller.org/ab-testing/sample-size.html}{online
calculator} and get the sample size. Here is an illustration of the
sample size online calculator.

    \includegraphics{sample_size.png}

    Once we have calulated the sample size, we can convert it to pageviews.
Here is a summary of \textbf{pageviews} required for each evaluation
metrics to achieve targeted statistical power.

    Gross Conversion - Baseline conversion: 20.625\% - Minimum Detectable
effect: 1\% - Alpha: 5\% - Power: 80\% - Sample size (calulated using
the online calculator): 25,835 enrollments/group - Number of groups: 2
(experiment and control) - Total sample size = 25835*2 = 51,670
enrollments - Clicks/pageviews: 3200/40,000 - Pageview required =
enrollments/(clicks/pageviews)=51670/(3200/40000) = 645,875

    Retention - Baseline conversion: 53\% - Minimum Detectable effect: 1\% -
Alpha: 5\% - Power: 80\% - Sample size (calulated using the online
calculator): 39,155 enrollments/group - Number of groups: 2 (experiment
and control) - Total sample size = 39155*2 = 78,230 enrollments -
Enrollments/pageviews: 660/40,000 - Pageview required =
enrollments/(enrollment/pageviews)=78230/(660/40000) = 4,741,212

    Net Conversion - Baseline conversion: 10.93125\% - Minimum Detectable
effect: 0.75\% - Alpha: 5\% - Power: 80\% - Sample size (calulated using
the online calculator): 27,413 enrollments/group - Number of groups: 2
(experiment and control) - Total sample size = 27413*2 = 54,826
enrollments - Clicks/pageviews: 3200/40,000 - Pageview required =
enrollments/(clicks/pageviews)=54826/(3200/40000) = 685,325

    The required pageviews is the maximum of pageviews for all three
evaluation metrics, i.e., 4,741,212 pageviews.

    \hypertarget{choose-duration-vs-exposure}{%
\subsubsection{5.2 Choose duration vs
exposure}\label{choose-duration-vs-exposure}}

    Let's assume we direct 80\% of the web traffic to this experiment. Given
40,000 pageviews per day, the experiment would take 148 day to achieve
the required 4,741,212 pageviews. This would take too long and is not
practical, since We have to wait for 5 months for the experimental
results in order to make a business decision. Therefore we have to drop
rentention as an evaluation metric. Now we are left with two evaluation
metrics; this reduced the number of pageviews required to 685,325. The
experiment will takes about 22 days to achieve the required pageviews.

    \hypertarget{experimental-analysis}{%
\subsection{6. Experimental Analysis}\label{experimental-analysis}}

    \hypertarget{sanity-checks}{%
\subsubsection{6.1 Sanity checks}\label{sanity-checks}}

    Let's check if the \textbf{invariant} metrics are randomly split between
the control and experiment group. We can assume a binomial distribution
here; confidence level is set at 95\%. If sanity check fails, look at
the day-by-day data to see if they are any anomalies that can give us
some insights into what might be the problem. Do not proceed to the rest
of the analysis unless all sanity checks pass.

    The data are provided by Udacity as two spreadsheets (control and
experimental groups). We'll load them into pandas dataframe.

    \begin{tcolorbox}[breakable, size=fbox, boxrule=1pt, pad at break*=1mm,colback=cellbackground, colframe=cellborder]
\prompt{In}{incolor}{7}{\boxspacing}
\begin{Verbatim}[commandchars=\\\{\}]
\PY{n}{con} \PY{o}{=} \PY{n}{pd}\PY{o}{.}\PY{n}{read\PYZus{}csv}\PY{p}{(}\PY{l+s+s1}{\PYZsq{}}\PY{l+s+s1}{control.csv}\PY{l+s+s1}{\PYZsq{}}\PY{p}{)}
\PY{n}{exp} \PY{o}{=} \PY{n}{pd}\PY{o}{.}\PY{n}{read\PYZus{}csv}\PY{p}{(}\PY{l+s+s1}{\PYZsq{}}\PY{l+s+s1}{experiment.csv}\PY{l+s+s1}{\PYZsq{}}\PY{p}{)}
\PY{n+nb}{print}\PY{p}{(}\PY{n}{con}\PY{p}{)}
\end{Verbatim}
\end{tcolorbox}

    \begin{Verbatim}[commandchars=\\\{\}]
           Date  Pageviews  Clicks  Enrollments  Payments
0   Sat, Oct 11       7723     687        134.0      70.0
1   Sun, Oct 12       9102     779        147.0      70.0
2   Mon, Oct 13      10511     909        167.0      95.0
3   Tue, Oct 14       9871     836        156.0     105.0
4   Wed, Oct 15      10014     837        163.0      64.0
5   Thu, Oct 16       9670     823        138.0      82.0
6   Fri, Oct 17       9008     748        146.0      76.0
7   Sat, Oct 18       7434     632        110.0      70.0
8   Sun, Oct 19       8459     691        131.0      60.0
9   Mon, Oct 20      10667     861        165.0      97.0
10  Tue, Oct 21      10660     867        196.0     105.0
11  Wed, Oct 22       9947     838        162.0      92.0
12  Thu, Oct 23       8324     665        127.0      56.0
13  Fri, Oct 24       9434     673        220.0     122.0
14  Sat, Oct 25       8687     691        176.0     128.0
15  Sun, Oct 26       8896     708        161.0     104.0
16  Mon, Oct 27       9535     759        233.0     124.0
17  Tue, Oct 28       9363     736        154.0      91.0
18  Wed, Oct 29       9327     739        196.0      86.0
19  Thu, Oct 30       9345     734        167.0      75.0
20  Fri, Oct 31       8890     706        174.0     101.0
21   Sat, Nov 1       8460     681        156.0      93.0
22   Sun, Nov 2       8836     693        206.0      67.0
23   Mon, Nov 3       9437     788          NaN       NaN
24   Tue, Nov 4       9420     781          NaN       NaN
25   Wed, Nov 5       9570     805          NaN       NaN
26   Thu, Nov 6       9921     830          NaN       NaN
27   Fri, Nov 7       9424     781          NaN       NaN
28   Sat, Nov 8       9010     756          NaN       NaN
29   Sun, Nov 9       9656     825          NaN       NaN
30  Mon, Nov 10      10419     874          NaN       NaN
31  Tue, Nov 11       9880     830          NaN       NaN
32  Wed, Nov 12      10134     801          NaN       NaN
33  Thu, Nov 13       9717     814          NaN       NaN
34  Fri, Nov 14       9192     735          NaN       NaN
35  Sat, Nov 15       8630     743          NaN       NaN
36  Sun, Nov 16       8970     722          NaN       NaN
    \end{Verbatim}

    The meaning of each column is: - Pageviews: Number of unique cookies to
view the course overview page that day. - Clicks: Number of unique
cookies to click the course overview page that day. - Enrollments:
Number of user-ids to enroll in the free trial that day. - Payments:
Number of user-ids who who enrolled on that day to remain enrolled for
14 days and thus make a payment. (Note that the date for this column is
the start date, that is, the date of enrollment, rather than the date of
the payment. The payment happened 14 days later. Because of this, the
enrollments and payments are tracked for 14 fewer days than the other
columns.)

    Our three invariant metrics are cookies (pageviews), clicks and CTP
(clicks/pageviews). Let's do sanity checks on them one by one.

    \begin{itemize}
\tightlist
\item
  Pageview: number of unique cookies to view the course overview page
\end{itemize}

    \begin{tcolorbox}[breakable, size=fbox, boxrule=1pt, pad at break*=1mm,colback=cellbackground, colframe=cellborder]
\prompt{In}{incolor}{8}{\boxspacing}
\begin{Verbatim}[commandchars=\\\{\}]
\PY{n}{cookies\PYZus{}con} \PY{o}{=} \PY{n}{con}\PY{p}{[}\PY{l+s+s1}{\PYZsq{}}\PY{l+s+s1}{Pageviews}\PY{l+s+s1}{\PYZsq{}}\PY{p}{]}\PY{o}{.}\PY{n}{sum}\PY{p}{(}\PY{p}{)}
\PY{n}{cookies\PYZus{}exp} \PY{o}{=} \PY{n}{exp}\PY{p}{[}\PY{l+s+s1}{\PYZsq{}}\PY{l+s+s1}{Pageviews}\PY{l+s+s1}{\PYZsq{}}\PY{p}{]}\PY{o}{.}\PY{n}{sum}\PY{p}{(}\PY{p}{)}
\PY{n}{cookies\PYZus{}total} \PY{o}{=} \PY{n}{cookies\PYZus{}con} \PY{o}{+} \PY{n}{cookies\PYZus{}exp}
\PY{n+nb}{print}\PY{p}{(}\PY{l+s+s1}{\PYZsq{}}\PY{l+s+s1}{number of unique cookies to view the course page in the control group:}\PY{l+s+s1}{\PYZsq{}}\PY{p}{,} \PY{n}{cookies\PYZus{}con}\PY{p}{)}
\PY{n+nb}{print}\PY{p}{(}\PY{l+s+s1}{\PYZsq{}}\PY{l+s+s1}{number of unique cookies to view the course page in the experiment group:}\PY{l+s+s1}{\PYZsq{}}\PY{p}{,} \PY{n}{cookies\PYZus{}exp}\PY{p}{)}
\end{Verbatim}
\end{tcolorbox}

    \begin{Verbatim}[commandchars=\\\{\}]
number of unique cookies to view the course page in the control group: 345543
number of unique cookies to view the course page in the experiment group: 344660
    \end{Verbatim}

    We eyeballed the numbers and they look close to each other. We expect
the amount of pageviews in the control and experiment groups to be split
even, roughtly 50\% each. The expected probability of a sample getting
assigned to the control group is 0.5. Next we alculate the standard
deviation and confidence interval and check if the observed probability
p\_hat is within the confidence interval.

    \begin{tcolorbox}[breakable, size=fbox, boxrule=1pt, pad at break*=1mm,colback=cellbackground, colframe=cellborder]
\prompt{In}{incolor}{9}{\boxspacing}
\begin{Verbatim}[commandchars=\\\{\}]
\PY{n}{p} \PY{o}{=} \PY{l+m+mf}{0.5}
\PY{n}{p\PYZus{}hat} \PY{o}{=} \PY{n}{cookies\PYZus{}con}\PY{o}{/}\PY{n}{cookies\PYZus{}total}
\PY{n}{sd} \PY{o}{=} \PY{n}{math}\PY{o}{.}\PY{n}{sqrt}\PY{p}{(}\PY{n}{p\PYZus{}hat}\PY{o}{*}\PY{p}{(}\PY{l+m+mi}{1}\PY{o}{\PYZhy{}}\PY{n}{p\PYZus{}hat}\PY{p}{)}\PY{o}{/}\PY{n}{cookies\PYZus{}total}\PY{p}{)}
\PY{n}{m} \PY{o}{=} \PY{l+m+mf}{1.96}\PY{o}{*}\PY{n}{sd}
\PY{n}{lower\PYZus{}bound}\PY{p}{,} \PY{n}{upper\PYZus{}bound} \PY{o}{=} \PY{n+nb}{round}\PY{p}{(}\PY{n}{p}\PY{o}{+}\PY{n}{m}\PY{p}{,}\PY{l+m+mi}{4}\PY{p}{)}\PY{p}{,} \PY{n+nb}{round}\PY{p}{(}\PY{n}{p}\PY{o}{\PYZhy{}}\PY{n}{m}\PY{p}{,}\PY{l+m+mi}{4}\PY{p}{)}
\PY{n+nb}{print}\PY{p}{(}\PY{l+s+s1}{\PYZsq{}}\PY{l+s+s1}{Obersved p\PYZus{}hat}\PY{l+s+s1}{\PYZsq{}}\PY{p}{,} \PY{n+nb}{round}\PY{p}{(}\PY{n}{p\PYZus{}hat}\PY{p}{,}\PY{l+m+mi}{4}\PY{p}{)}\PY{p}{)}
\PY{n+nb}{print}\PY{p}{(}\PY{l+s+s1}{\PYZsq{}}\PY{l+s+s1}{The confidence interval is (}\PY{l+s+si}{\PYZob{}\PYZcb{}}\PY{l+s+s1}{,}\PY{l+s+si}{\PYZob{}\PYZcb{}}\PY{l+s+s1}{)}\PY{l+s+s1}{\PYZsq{}}\PY{o}{.}\PY{n}{format}\PY{p}{(}\PY{n}{lower\PYZus{}bound}\PY{p}{,}\PY{n}{upper\PYZus{}bound}\PY{p}{)}\PY{p}{)}
\end{Verbatim}
\end{tcolorbox}

    \begin{Verbatim}[commandchars=\\\{\}]
Obersved p\_hat 0.5006
The confidence interval is (0.5012,0.4988)
    \end{Verbatim}

    p\_hat is within the confidence interval, which means the control and
experiments are randomly split. The invariant metric cookies passed the
sanity check.

    \begin{itemize}
\tightlist
\item
  Clicks: Number of unique cookies to click the course overview page
  that day.
\end{itemize}

    \begin{tcolorbox}[breakable, size=fbox, boxrule=1pt, pad at break*=1mm,colback=cellbackground, colframe=cellborder]
\prompt{In}{incolor}{10}{\boxspacing}
\begin{Verbatim}[commandchars=\\\{\}]
\PY{n}{clicks\PYZus{}con} \PY{o}{=} \PY{n}{con}\PY{p}{[}\PY{l+s+s1}{\PYZsq{}}\PY{l+s+s1}{Clicks}\PY{l+s+s1}{\PYZsq{}}\PY{p}{]}\PY{o}{.}\PY{n}{sum}\PY{p}{(}\PY{p}{)}
\PY{n}{clicks\PYZus{}exp} \PY{o}{=} \PY{n}{exp}\PY{p}{[}\PY{l+s+s1}{\PYZsq{}}\PY{l+s+s1}{Clicks}\PY{l+s+s1}{\PYZsq{}}\PY{p}{]}\PY{o}{.}\PY{n}{sum}\PY{p}{(}\PY{p}{)}
\PY{n}{clicks\PYZus{}total} \PY{o}{=} \PY{n}{clicks\PYZus{}con} \PY{o}{+} \PY{n}{clicks\PYZus{}exp}
\PY{n+nb}{print}\PY{p}{(}\PY{l+s+s1}{\PYZsq{}}\PY{l+s+s1}{number of unique cookies to click the course page in the control group:}\PY{l+s+s1}{\PYZsq{}}\PY{p}{,} \PY{n}{clicks\PYZus{}con}\PY{p}{)}
\PY{n+nb}{print}\PY{p}{(}\PY{l+s+s1}{\PYZsq{}}\PY{l+s+s1}{number of unique cookies to click the course page in the experiment group:}\PY{l+s+s1}{\PYZsq{}}\PY{p}{,} \PY{n}{clicks\PYZus{}exp}\PY{p}{)}
\end{Verbatim}
\end{tcolorbox}

    \begin{Verbatim}[commandchars=\\\{\}]
number of unique cookies to click the course page in the control group: 28378
number of unique cookies to click the course page in the experiment group: 28325
    \end{Verbatim}

    \begin{tcolorbox}[breakable, size=fbox, boxrule=1pt, pad at break*=1mm,colback=cellbackground, colframe=cellborder]
\prompt{In}{incolor}{11}{\boxspacing}
\begin{Verbatim}[commandchars=\\\{\}]
\PY{n}{p\PYZus{}hat} \PY{o}{=} \PY{n}{clicks\PYZus{}con}\PY{o}{/}\PY{n}{clicks\PYZus{}total}
\PY{n}{sd} \PY{o}{=} \PY{n}{math}\PY{o}{.}\PY{n}{sqrt}\PY{p}{(}\PY{n}{p\PYZus{}hat}\PY{o}{*}\PY{p}{(}\PY{l+m+mi}{1}\PY{o}{\PYZhy{}}\PY{n}{p\PYZus{}hat}\PY{p}{)}\PY{o}{/}\PY{n}{clicks\PYZus{}total}\PY{p}{)}
\PY{n}{m} \PY{o}{=} \PY{l+m+mf}{1.96}\PY{o}{*}\PY{n}{sd}
\PY{n}{lower\PYZus{}bound}\PY{p}{,} \PY{n}{upper\PYZus{}bound} \PY{o}{=} \PY{n+nb}{round}\PY{p}{(}\PY{n}{p}\PY{o}{+}\PY{n}{m}\PY{p}{,}\PY{l+m+mi}{4}\PY{p}{)}\PY{p}{,} \PY{n+nb}{round}\PY{p}{(}\PY{n}{p}\PY{o}{\PYZhy{}}\PY{n}{m}\PY{p}{,}\PY{l+m+mi}{4}\PY{p}{)}
\PY{n+nb}{print}\PY{p}{(}\PY{l+s+s1}{\PYZsq{}}\PY{l+s+s1}{Observed p\PYZus{}hat}\PY{l+s+s1}{\PYZsq{}}\PY{p}{,} \PY{n+nb}{round}\PY{p}{(}\PY{n}{p\PYZus{}hat}\PY{p}{,}\PY{l+m+mi}{4}\PY{p}{)}\PY{p}{)}
\PY{n+nb}{print}\PY{p}{(}\PY{l+s+s1}{\PYZsq{}}\PY{l+s+s1}{The confidence interval is (}\PY{l+s+si}{\PYZob{}\PYZcb{}}\PY{l+s+s1}{,}\PY{l+s+si}{\PYZob{}\PYZcb{}}\PY{l+s+s1}{)}\PY{l+s+s1}{\PYZsq{}}\PY{o}{.}\PY{n}{format}\PY{p}{(}\PY{n}{lower\PYZus{}bound}\PY{p}{,}\PY{n}{upper\PYZus{}bound}\PY{p}{)}\PY{p}{)}
\end{Verbatim}
\end{tcolorbox}

    \begin{Verbatim}[commandchars=\\\{\}]
Observed p\_hat 0.5005
The confidence interval is (0.5041,0.4959)
    \end{Verbatim}

    \begin{itemize}
\tightlist
\item
  CTP (clicks/cookies)
\end{itemize}

    \begin{tcolorbox}[breakable, size=fbox, boxrule=1pt, pad at break*=1mm,colback=cellbackground, colframe=cellborder]
\prompt{In}{incolor}{12}{\boxspacing}
\begin{Verbatim}[commandchars=\\\{\}]
\PY{n}{ctp\PYZus{}con} \PY{o}{=} \PY{n}{clicks\PYZus{}con}\PY{o}{/}\PY{n}{cookies\PYZus{}con}
\PY{n}{ctp\PYZus{}exp} \PY{o}{=} \PY{n}{clicks\PYZus{}exp}\PY{o}{/}\PY{n}{cookies\PYZus{}exp}
\PY{n}{d\PYZus{}hat} \PY{o}{=} \PY{n}{ctp\PYZus{}exp} \PY{o}{\PYZhy{}} \PY{n}{ctp\PYZus{}con}
\PY{n}{p\PYZus{}pooled} \PY{o}{=} \PY{n}{clicks\PYZus{}total}\PY{o}{/}\PY{n}{cookies\PYZus{}total}
\PY{n}{sd\PYZus{}pooled} \PY{o}{=} \PY{n}{math}\PY{o}{.}\PY{n}{sqrt}\PY{p}{(}\PY{n}{p\PYZus{}pooled}\PY{o}{*}\PY{p}{(}\PY{l+m+mi}{1}\PY{o}{\PYZhy{}}\PY{n}{p\PYZus{}pooled}\PY{p}{)}\PY{o}{*}\PY{p}{(}\PY{l+m+mi}{1}\PY{o}{/}\PY{n}{cookies\PYZus{}con}\PY{o}{+}\PY{l+m+mi}{1}\PY{o}{/}\PY{n}{cookies\PYZus{}exp}\PY{p}{)}\PY{p}{)}
\PY{n}{m} \PY{o}{=} \PY{l+m+mf}{1.96}\PY{o}{*}\PY{n}{sd\PYZus{}pooled}
\PY{n}{lower\PYZus{}bound}\PY{p}{,} \PY{n}{upper\PYZus{}bound} \PY{o}{=} \PY{n+nb}{round}\PY{p}{(}\PY{l+m+mi}{0}\PY{o}{+}\PY{n}{m}\PY{p}{,}\PY{l+m+mi}{4}\PY{p}{)}\PY{p}{,} \PY{n+nb}{round}\PY{p}{(}\PY{l+m+mi}{0}\PY{o}{\PYZhy{}}\PY{n}{m}\PY{p}{,}\PY{l+m+mi}{4}\PY{p}{)}
\PY{n+nb}{print}\PY{p}{(}\PY{l+s+s1}{\PYZsq{}}\PY{l+s+s1}{Oberved difference}\PY{l+s+s1}{\PYZsq{}}\PY{p}{,} \PY{n+nb}{round}\PY{p}{(}\PY{n}{d\PYZus{}hat}\PY{p}{,}\PY{l+m+mi}{4}\PY{p}{)}\PY{p}{)}
\PY{n+nb}{print}\PY{p}{(}\PY{l+s+s1}{\PYZsq{}}\PY{l+s+s1}{The confidence interval is (}\PY{l+s+si}{\PYZob{}\PYZcb{}}\PY{l+s+s1}{,}\PY{l+s+si}{\PYZob{}\PYZcb{}}\PY{l+s+s1}{)}\PY{l+s+s1}{\PYZsq{}}\PY{o}{.}\PY{n}{format}\PY{p}{(}\PY{n}{lower\PYZus{}bound}\PY{p}{,}\PY{n}{upper\PYZus{}bound}\PY{p}{)}\PY{p}{)}
\end{Verbatim}
\end{tcolorbox}

    \begin{Verbatim}[commandchars=\\\{\}]
Oberved difference 0.0001
The confidence interval is (0.0013,-0.0013)
    \end{Verbatim}

    We can see they all three invariant metrics passed sanity checks.

    \hypertarget{effec-size-tests}{%
\subsubsection{6.2 Effec size tests}\label{effec-size-tests}}

    For \textbf{evaluation} metrics, give a 95\% confidence level for the
difference between the experiment and control groups, calculate its
confidence interval. Check if each metric is statistically siginificant
and practically significant. A metric is statistically significant if
the confidence interval does not include 0 (i.e., we can be confident
there was a change); and it is practically significant if the confidence
interval does not include the practical significance boundary Dmin
(i.e., you can be confident there is a change that matters to the
business.)

    \begin{itemize}
\tightlist
\item
  Gross conversion
\end{itemize}

    Note: the spreadsheet provided by Udacity contains data for cookies and
clicks for 37 days, and enrollments and payments for 23 days. So we we
work with gross conversion and net conversion, we will only use the data
for 23 days. We expect the gross conversion drops in the experimental
group.

    \begin{tcolorbox}[breakable, size=fbox, boxrule=1pt, pad at break*=1mm,colback=cellbackground, colframe=cellborder]
\prompt{In}{incolor}{13}{\boxspacing}
\begin{Verbatim}[commandchars=\\\{\}]
\PY{c+c1}{\PYZsh{} Count the total clicks for the 23 days where the \PYZsq{}Enrollments\PYZsq{} column is not NaN}
\PY{n}{clicks\PYZus{}con} \PY{o}{=} \PY{n}{con}\PY{p}{[}\PY{l+s+s1}{\PYZsq{}}\PY{l+s+s1}{Clicks}\PY{l+s+s1}{\PYZsq{}}\PY{p}{]}\PY{o}{.}\PY{n}{loc}\PY{p}{[}\PY{n}{con}\PY{p}{[}\PY{l+s+s1}{\PYZsq{}}\PY{l+s+s1}{Enrollments}\PY{l+s+s1}{\PYZsq{}}\PY{p}{]}\PY{o}{.}\PY{n}{notnull}\PY{p}{(}\PY{p}{)}\PY{p}{]}\PY{o}{.}\PY{n}{sum}\PY{p}{(}\PY{p}{)}
\PY{n}{clicks\PYZus{}exp} \PY{o}{=} \PY{n}{exp}\PY{p}{[}\PY{l+s+s1}{\PYZsq{}}\PY{l+s+s1}{Clicks}\PY{l+s+s1}{\PYZsq{}}\PY{p}{]}\PY{o}{.}\PY{n}{loc}\PY{p}{[}\PY{n}{exp}\PY{p}{[}\PY{l+s+s1}{\PYZsq{}}\PY{l+s+s1}{Enrollments}\PY{l+s+s1}{\PYZsq{}}\PY{p}{]}\PY{o}{.}\PY{n}{notnull}\PY{p}{(}\PY{p}{)}\PY{p}{]}\PY{o}{.}\PY{n}{sum}\PY{p}{(}\PY{p}{)}
\end{Verbatim}
\end{tcolorbox}

    \begin{tcolorbox}[breakable, size=fbox, boxrule=1pt, pad at break*=1mm,colback=cellbackground, colframe=cellborder]
\prompt{In}{incolor}{36}{\boxspacing}
\begin{Verbatim}[commandchars=\\\{\}]
\PY{n}{enrollments\PYZus{}con} \PY{o}{=} \PY{n}{con}\PY{p}{[}\PY{l+s+s1}{\PYZsq{}}\PY{l+s+s1}{Enrollments}\PY{l+s+s1}{\PYZsq{}}\PY{p}{]}\PY{o}{.}\PY{n}{sum}\PY{p}{(}\PY{p}{)}
\PY{n}{enrollments\PYZus{}exp} \PY{o}{=} \PY{n}{exp}\PY{p}{[}\PY{l+s+s1}{\PYZsq{}}\PY{l+s+s1}{Enrollments}\PY{l+s+s1}{\PYZsq{}}\PY{p}{]}\PY{o}{.}\PY{n}{sum}\PY{p}{(}\PY{p}{)}
\PY{n}{g\PYZus{}con} \PY{o}{=} \PY{n}{enrollments\PYZus{}con}\PY{o}{/}\PY{n}{clicks\PYZus{}con}
\PY{n}{g\PYZus{}exp} \PY{o}{=} \PY{n}{enrollments\PYZus{}exp}\PY{o}{/}\PY{n}{clicks\PYZus{}exp}
\PY{n}{g\PYZus{}diff} \PY{o}{=} \PY{n}{g\PYZus{}exp}\PY{o}{\PYZhy{}}\PY{n}{g\PYZus{}con}
\PY{n}{g\PYZus{}pooled} \PY{o}{=} \PY{p}{(}\PY{n}{enrollments\PYZus{}con}\PY{o}{+}\PY{n}{enrollments\PYZus{}exp}\PY{p}{)}\PY{o}{/}\PY{p}{(}\PY{n}{clicks\PYZus{}con}\PY{o}{+}\PY{n}{clicks\PYZus{}exp}\PY{p}{)}
\PY{n}{g\PYZus{}sd} \PY{o}{=} \PY{n}{math}\PY{o}{.}\PY{n}{sqrt}\PY{p}{(}\PY{n}{g\PYZus{}pooled}\PY{o}{*}\PY{p}{(}\PY{l+m+mi}{1}\PY{o}{\PYZhy{}}\PY{n}{g\PYZus{}pooled}\PY{p}{)}\PY{o}{*}\PY{p}{(}\PY{l+m+mi}{1}\PY{o}{/}\PY{n}{clicks\PYZus{}con}\PY{o}{+}\PY{l+m+mi}{1}\PY{o}{/}\PY{n}{clicks\PYZus{}exp}\PY{p}{)}\PY{p}{)}
\PY{n}{m} \PY{o}{=} \PY{l+m+mf}{1.96}\PY{o}{*}\PY{n}{g\PYZus{}sd}
\PY{n}{lower\PYZus{}bound}\PY{p}{,} \PY{n}{upper\PYZus{}bound} \PY{o}{=} \PY{n+nb}{round}\PY{p}{(}\PY{n}{g\PYZus{}diff}\PY{o}{\PYZhy{}}\PY{n}{m}\PY{p}{,}\PY{l+m+mi}{4}\PY{p}{)}\PY{p}{,} \PY{n+nb}{round}\PY{p}{(}\PY{n}{g\PYZus{}diff}\PY{o}{+}\PY{n}{m}\PY{p}{,}\PY{l+m+mi}{4}\PY{p}{)}
\PY{n+nb}{print}\PY{p}{(}\PY{l+s+s1}{\PYZsq{}}\PY{l+s+s1}{The change due to the experiment is }\PY{l+s+si}{\PYZob{}\PYZcb{}}\PY{l+s+s1}{\PYZsq{}}\PY{o}{.}\PY{n}{format}\PY{p}{(}\PY{n+nb}{round}\PY{p}{(}\PY{n}{g\PYZus{}diff}\PY{p}{,}\PY{l+m+mi}{4}\PY{p}{)}\PY{p}{)}\PY{p}{)}
\PY{n+nb}{print}\PY{p}{(}\PY{l+s+s1}{\PYZsq{}}\PY{l+s+s1}{The confidence interval is (}\PY{l+s+si}{\PYZob{}\PYZcb{}}\PY{l+s+s1}{,}\PY{l+s+si}{\PYZob{}\PYZcb{}}\PY{l+s+s1}{)}\PY{l+s+s1}{\PYZsq{}}\PY{o}{.}\PY{n}{format}\PY{p}{(}\PY{n}{lower\PYZus{}bound}\PY{p}{,}\PY{n}{upper\PYZus{}bound}\PY{p}{)}\PY{p}{)}
\end{Verbatim}
\end{tcolorbox}

    \begin{Verbatim}[commandchars=\\\{\}]
The change due to the experiment is -0.0206
The confidence interval is (-0.0291,-0.012)
    \end{Verbatim}

    The observed change is statistically significant since the confidence
interval does not include 0. And it is practically significant since the
confidence interval does not include the minimum detectable effect
-0.01.

    The change is both statiscally and practically signifcant. We have a
negative change of 2.06\%, which means the gross conversion rate in the
experiment group has dropped 2.06\% and this change is significant. This
matches with our expecation, as when customers click the ``Start free
trial'' button, they are asked how many hours they can dedicate to the
study per week; this change should make customers think twice before
they go ahead and enroll, hence the decrease gross conversion rate.

    \begin{itemize}
\tightlist
\item
  Net conversion
\end{itemize}

    \begin{tcolorbox}[breakable, size=fbox, boxrule=1pt, pad at break*=1mm,colback=cellbackground, colframe=cellborder]
\prompt{In}{incolor}{15}{\boxspacing}
\begin{Verbatim}[commandchars=\\\{\}]
\PY{n}{payments\PYZus{}con} \PY{o}{=} \PY{n}{con}\PY{p}{[}\PY{l+s+s1}{\PYZsq{}}\PY{l+s+s1}{Payments}\PY{l+s+s1}{\PYZsq{}}\PY{p}{]}\PY{o}{.}\PY{n}{sum}\PY{p}{(}\PY{p}{)}
\PY{n}{payments\PYZus{}exp} \PY{o}{=} \PY{n}{exp}\PY{p}{[}\PY{l+s+s1}{\PYZsq{}}\PY{l+s+s1}{Payments}\PY{l+s+s1}{\PYZsq{}}\PY{p}{]}\PY{o}{.}\PY{n}{sum}\PY{p}{(}\PY{p}{)}
\PY{n}{n\PYZus{}con} \PY{o}{=} \PY{n}{payments\PYZus{}con}\PY{o}{/}\PY{n}{clicks\PYZus{}con}
\PY{n}{n\PYZus{}exp} \PY{o}{=} \PY{n}{payments\PYZus{}exp}\PY{o}{/}\PY{n}{clicks\PYZus{}exp}
\PY{n}{n\PYZus{}diff} \PY{o}{=} \PY{n}{n\PYZus{}exp}\PY{o}{\PYZhy{}}\PY{n}{n\PYZus{}con}
\PY{n}{n\PYZus{}pooled} \PY{o}{=} \PY{p}{(}\PY{n}{payments\PYZus{}con}\PY{o}{+}\PY{n}{payments\PYZus{}exp}\PY{p}{)}\PY{o}{/}\PY{p}{(}\PY{n}{clicks\PYZus{}con}\PY{o}{+}\PY{n}{clicks\PYZus{}exp}\PY{p}{)}
\PY{n}{n\PYZus{}sd} \PY{o}{=} \PY{n}{math}\PY{o}{.}\PY{n}{sqrt}\PY{p}{(}\PY{n}{n\PYZus{}pooled}\PY{o}{*}\PY{p}{(}\PY{l+m+mi}{1}\PY{o}{\PYZhy{}}\PY{n}{n\PYZus{}pooled}\PY{p}{)}\PY{o}{*}\PY{p}{(}\PY{l+m+mi}{1}\PY{o}{/}\PY{n}{clicks\PYZus{}con}\PY{o}{+}\PY{l+m+mi}{1}\PY{o}{/}\PY{n}{clicks\PYZus{}exp}\PY{p}{)}\PY{p}{)}
\PY{n}{m} \PY{o}{=} \PY{l+m+mf}{1.96}\PY{o}{*}\PY{n}{n\PYZus{}sd}
\PY{n}{lower\PYZus{}bound}\PY{p}{,} \PY{n}{upper\PYZus{}bound} \PY{o}{=} \PY{n+nb}{round}\PY{p}{(}\PY{n}{n\PYZus{}diff}\PY{o}{\PYZhy{}}\PY{n}{m}\PY{p}{,}\PY{l+m+mi}{4}\PY{p}{)}\PY{p}{,} \PY{n+nb}{round}\PY{p}{(}\PY{n}{n\PYZus{}diff}\PY{o}{+}\PY{n}{m}\PY{p}{,}\PY{l+m+mi}{4}\PY{p}{)}
\PY{n+nb}{print}\PY{p}{(}\PY{l+s+s1}{\PYZsq{}}\PY{l+s+s1}{The change due to the experiment is }\PY{l+s+si}{\PYZob{}\PYZcb{}}\PY{l+s+s1}{\PYZsq{}}\PY{o}{.}\PY{n}{format}\PY{p}{(}\PY{n+nb}{round}\PY{p}{(}\PY{n}{n\PYZus{}diff}\PY{p}{,}\PY{l+m+mi}{4}\PY{p}{)}\PY{p}{)}\PY{p}{)}
\PY{n+nb}{print}\PY{p}{(}\PY{l+s+s1}{\PYZsq{}}\PY{l+s+s1}{The confidence interval is (}\PY{l+s+si}{\PYZob{}\PYZcb{}}\PY{l+s+s1}{,}\PY{l+s+si}{\PYZob{}\PYZcb{}}\PY{l+s+s1}{)}\PY{l+s+s1}{\PYZsq{}}\PY{o}{.}\PY{n}{format}\PY{p}{(}\PY{n}{lower\PYZus{}bound}\PY{p}{,}\PY{n}{upper\PYZus{}bound}\PY{p}{)}\PY{p}{)}
\PY{n+nb}{print}\PY{p}{(}\PY{l+s+s1}{\PYZsq{}}\PY{l+s+s1}{The observed change is statistically significant if the confidence interval does not include 0. And it is practically significant if it does not include the minimum detectable effect \PYZhy{}0.0075.}\PY{l+s+s1}{\PYZsq{}}\PY{p}{)}
\end{Verbatim}
\end{tcolorbox}

    \begin{Verbatim}[commandchars=\\\{\}]
The change due to the experiment is -0.0049
The confidence interval is (-0.0116,0.0019)
The observed change is statistically significant if the confidence interval does
not include 0. And it is practically significant if it does not include the
minimum detectable effect -0.0075.
    \end{Verbatim}

    We got a very small change of -0.49\%, which is neither statistically
significant, nor practically significant.

    \hypertarget{sign-tests}{%
\subsubsection{6.3 Sign tests}\label{sign-tests}}

    For each \textbf{evalution} metric, let's count day-by-day how many days
in the experiment group is lower than the control group out of 23 days,
and this is the number of successes for our binomial model. We then
cauculate the p-value and compare it with alpha.

    \begin{tcolorbox}[breakable, size=fbox, boxrule=1pt, pad at break*=1mm,colback=cellbackground, colframe=cellborder]
\prompt{In}{incolor}{24}{\boxspacing}
\begin{Verbatim}[commandchars=\\\{\}]
\PY{c+c1}{\PYZsh{} Let\PYZsq{}s merge the two dataframes since we need to compare day\PYZhy{}to\PYZhy{}day gross conversion and net conversion in control and expeirment group}
\PY{n}{combine} \PY{o}{=} \PY{n}{con}\PY{o}{.}\PY{n}{join}\PY{p}{(}\PY{n}{exp}\PY{p}{,} \PY{n}{how}\PY{o}{=}\PY{l+s+s1}{\PYZsq{}}\PY{l+s+s1}{inner}\PY{l+s+s1}{\PYZsq{}}\PY{p}{,}\PY{n}{lsuffix}\PY{o}{=}\PY{l+s+s1}{\PYZsq{}}\PY{l+s+s1}{\PYZus{}con}\PY{l+s+s1}{\PYZsq{}}\PY{p}{,} \PY{n}{rsuffix}\PY{o}{=}\PY{l+s+s1}{\PYZsq{}}\PY{l+s+s1}{\PYZus{}exp}\PY{l+s+s1}{\PYZsq{}}\PY{p}{)}
\PY{n}{combine}
\end{Verbatim}
\end{tcolorbox}

            \begin{tcolorbox}[breakable, size=fbox, boxrule=.5pt, pad at break*=1mm, opacityfill=0]
\prompt{Out}{outcolor}{24}{\boxspacing}
\begin{Verbatim}[commandchars=\\\{\}]
       Date\_con  Pageviews\_con  Clicks\_con  Enrollments\_con  Payments\_con  \textbackslash{}
0   Sat, Oct 11           7723         687            134.0          70.0
1   Sun, Oct 12           9102         779            147.0          70.0
2   Mon, Oct 13          10511         909            167.0          95.0
3   Tue, Oct 14           9871         836            156.0         105.0
4   Wed, Oct 15          10014         837            163.0          64.0
5   Thu, Oct 16           9670         823            138.0          82.0
6   Fri, Oct 17           9008         748            146.0          76.0
7   Sat, Oct 18           7434         632            110.0          70.0
8   Sun, Oct 19           8459         691            131.0          60.0
9   Mon, Oct 20          10667         861            165.0          97.0
10  Tue, Oct 21          10660         867            196.0         105.0
11  Wed, Oct 22           9947         838            162.0          92.0
12  Thu, Oct 23           8324         665            127.0          56.0
13  Fri, Oct 24           9434         673            220.0         122.0
14  Sat, Oct 25           8687         691            176.0         128.0
15  Sun, Oct 26           8896         708            161.0         104.0
16  Mon, Oct 27           9535         759            233.0         124.0
17  Tue, Oct 28           9363         736            154.0          91.0
18  Wed, Oct 29           9327         739            196.0          86.0
19  Thu, Oct 30           9345         734            167.0          75.0
20  Fri, Oct 31           8890         706            174.0         101.0
21   Sat, Nov 1           8460         681            156.0          93.0
22   Sun, Nov 2           8836         693            206.0          67.0
23   Mon, Nov 3           9437         788              NaN           NaN
24   Tue, Nov 4           9420         781              NaN           NaN
25   Wed, Nov 5           9570         805              NaN           NaN
26   Thu, Nov 6           9921         830              NaN           NaN
27   Fri, Nov 7           9424         781              NaN           NaN
28   Sat, Nov 8           9010         756              NaN           NaN
29   Sun, Nov 9           9656         825              NaN           NaN
30  Mon, Nov 10          10419         874              NaN           NaN
31  Tue, Nov 11           9880         830              NaN           NaN
32  Wed, Nov 12          10134         801              NaN           NaN
33  Thu, Nov 13           9717         814              NaN           NaN
34  Fri, Nov 14           9192         735              NaN           NaN
35  Sat, Nov 15           8630         743              NaN           NaN
36  Sun, Nov 16           8970         722              NaN           NaN

       Date\_exp  Pageviews\_exp  Clicks\_exp  Enrollments\_exp  Payments\_exp
0   Sat, Oct 11           7716         686            105.0          34.0
1   Sun, Oct 12           9288         785            116.0          91.0
2   Mon, Oct 13          10480         884            145.0          79.0
3   Tue, Oct 14           9867         827            138.0          92.0
4   Wed, Oct 15           9793         832            140.0          94.0
5   Thu, Oct 16           9500         788            129.0          61.0
6   Fri, Oct 17           9088         780            127.0          44.0
7   Sat, Oct 18           7664         652             94.0          62.0
8   Sun, Oct 19           8434         697            120.0          77.0
9   Mon, Oct 20          10496         860            153.0          98.0
10  Tue, Oct 21          10551         864            143.0          71.0
11  Wed, Oct 22           9737         801            128.0          70.0
12  Thu, Oct 23           8176         642            122.0          68.0
13  Fri, Oct 24           9402         697            194.0          94.0
14  Sat, Oct 25           8669         669            127.0          81.0
15  Sun, Oct 26           8881         693            153.0         101.0
16  Mon, Oct 27           9655         771            213.0         119.0
17  Tue, Oct 28           9396         736            162.0         120.0
18  Wed, Oct 29           9262         727            201.0          96.0
19  Thu, Oct 30           9308         728            207.0          67.0
20  Fri, Oct 31           8715         722            182.0         123.0
21   Sat, Nov 1           8448         695            142.0         100.0
22   Sun, Nov 2           8836         724            182.0         103.0
23   Mon, Nov 3           9359         789              NaN           NaN
24   Tue, Nov 4           9427         743              NaN           NaN
25   Wed, Nov 5           9633         808              NaN           NaN
26   Thu, Nov 6           9842         831              NaN           NaN
27   Fri, Nov 7           9272         767              NaN           NaN
28   Sat, Nov 8           8969         760              NaN           NaN
29   Sun, Nov 9           9697         850              NaN           NaN
30  Mon, Nov 10          10445         851              NaN           NaN
31  Tue, Nov 11           9931         831              NaN           NaN
32  Wed, Nov 12          10042         802              NaN           NaN
33  Thu, Nov 13           9721         829              NaN           NaN
34  Fri, Nov 14           9304         770              NaN           NaN
35  Sat, Nov 15           8668         724              NaN           NaN
36  Sun, Nov 16           8988         710              NaN           NaN
\end{Verbatim}
\end{tcolorbox}
        
    \begin{tcolorbox}[breakable, size=fbox, boxrule=1pt, pad at break*=1mm,colback=cellbackground, colframe=cellborder]
\prompt{In}{incolor}{25}{\boxspacing}
\begin{Verbatim}[commandchars=\\\{\}]
\PY{n}{combine}\PY{o}{.}\PY{n}{count}\PY{p}{(}\PY{p}{)}
\end{Verbatim}
\end{tcolorbox}

            \begin{tcolorbox}[breakable, size=fbox, boxrule=.5pt, pad at break*=1mm, opacityfill=0]
\prompt{Out}{outcolor}{25}{\boxspacing}
\begin{Verbatim}[commandchars=\\\{\}]
Date\_con           37
Pageviews\_con      37
Clicks\_con         37
Enrollments\_con    23
Payments\_con       23
Date\_exp           37
Pageviews\_exp      37
Clicks\_exp         37
Enrollments\_exp    23
Payments\_exp       23
dtype: int64
\end{Verbatim}
\end{tcolorbox}
        
    \begin{tcolorbox}[breakable, size=fbox, boxrule=1pt, pad at break*=1mm,colback=cellbackground, colframe=cellborder]
\prompt{In}{incolor}{26}{\boxspacing}
\begin{Verbatim}[commandchars=\\\{\}]
\PY{c+c1}{\PYZsh{} We only need rows where Enrollments and Payments are not NaN}
\PY{n}{combine} \PY{o}{=} \PY{n}{combine}\PY{o}{.}\PY{n}{loc}\PY{p}{[}\PY{n}{combine}\PY{p}{[}\PY{l+s+s1}{\PYZsq{}}\PY{l+s+s1}{Enrollments\PYZus{}con}\PY{l+s+s1}{\PYZsq{}}\PY{p}{]}\PY{o}{.}\PY{n}{notnull}\PY{p}{(}\PY{p}{)}\PY{p}{]}
\PY{n}{combine}\PY{o}{.}\PY{n}{count}\PY{p}{(}\PY{p}{)}
\end{Verbatim}
\end{tcolorbox}

            \begin{tcolorbox}[breakable, size=fbox, boxrule=.5pt, pad at break*=1mm, opacityfill=0]
\prompt{Out}{outcolor}{26}{\boxspacing}
\begin{Verbatim}[commandchars=\\\{\}]
Date\_con           23
Pageviews\_con      23
Clicks\_con         23
Enrollments\_con    23
Payments\_con       23
Date\_exp           23
Pageviews\_exp      23
Clicks\_exp         23
Enrollments\_exp    23
Payments\_exp       23
dtype: int64
\end{Verbatim}
\end{tcolorbox}
        
    \begin{tcolorbox}[breakable, size=fbox, boxrule=1pt, pad at break*=1mm,colback=cellbackground, colframe=cellborder]
\prompt{In}{incolor}{35}{\boxspacing}
\begin{Verbatim}[commandchars=\\\{\}]
\PY{c+c1}{\PYZsh{} We calcuate gross conversion and net conversion, then compare the day\PYZhy{}to\PYZhy{}day data from control and experiment group.}
\PY{c+c1}{\PYZsh{} If the experiment value is larger than the control group, the sign is 1, otherwise 0.}
\PY{n}{combine}\PY{o}{.}\PY{n}{loc}\PY{p}{[}\PY{p}{:}\PY{p}{,}\PY{l+s+s1}{\PYZsq{}}\PY{l+s+s1}{g\PYZus{}con}\PY{l+s+s1}{\PYZsq{}}\PY{p}{]} \PY{o}{=} \PY{n}{combine}\PY{p}{[}\PY{l+s+s1}{\PYZsq{}}\PY{l+s+s1}{Enrollments\PYZus{}con}\PY{l+s+s1}{\PYZsq{}}\PY{p}{]}\PY{o}{/}\PY{n}{combine}\PY{p}{[}\PY{l+s+s1}{\PYZsq{}}\PY{l+s+s1}{Clicks\PYZus{}con}\PY{l+s+s1}{\PYZsq{}}\PY{p}{]}
\PY{n}{combine}\PY{o}{.}\PY{n}{loc}\PY{p}{[}\PY{p}{:}\PY{p}{,}\PY{l+s+s1}{\PYZsq{}}\PY{l+s+s1}{g\PYZus{}exp}\PY{l+s+s1}{\PYZsq{}}\PY{p}{]} \PY{o}{=} \PY{n}{combine}\PY{p}{[}\PY{l+s+s1}{\PYZsq{}}\PY{l+s+s1}{Enrollments\PYZus{}exp}\PY{l+s+s1}{\PYZsq{}}\PY{p}{]}\PY{o}{/}\PY{n}{combine}\PY{p}{[}\PY{l+s+s1}{\PYZsq{}}\PY{l+s+s1}{Clicks\PYZus{}exp}\PY{l+s+s1}{\PYZsq{}}\PY{p}{]}
\PY{n}{combine}\PY{o}{.}\PY{n}{loc}\PY{p}{[}\PY{p}{:}\PY{p}{,}\PY{l+s+s1}{\PYZsq{}}\PY{l+s+s1}{g\PYZus{}sign}\PY{l+s+s1}{\PYZsq{}}\PY{p}{]} \PY{o}{=} \PY{n}{np}\PY{o}{.}\PY{n}{where}\PY{p}{(}\PY{n}{combine}\PY{p}{[}\PY{l+s+s1}{\PYZsq{}}\PY{l+s+s1}{g\PYZus{}exp}\PY{l+s+s1}{\PYZsq{}}\PY{p}{]}\PY{o}{\PYZgt{}}\PY{n}{combine}\PY{p}{[}\PY{l+s+s1}{\PYZsq{}}\PY{l+s+s1}{g\PYZus{}con}\PY{l+s+s1}{\PYZsq{}}\PY{p}{]}\PY{p}{,}\PY{l+m+mi}{1}\PY{p}{,}\PY{l+m+mi}{0}\PY{p}{)}
\PY{n}{combine}\PY{o}{.}\PY{n}{loc}\PY{p}{[}\PY{p}{:}\PY{p}{,}\PY{l+s+s1}{\PYZsq{}}\PY{l+s+s1}{n\PYZus{}con}\PY{l+s+s1}{\PYZsq{}}\PY{p}{]} \PY{o}{=} \PY{n}{combine}\PY{p}{[}\PY{l+s+s1}{\PYZsq{}}\PY{l+s+s1}{Payments\PYZus{}con}\PY{l+s+s1}{\PYZsq{}}\PY{p}{]}\PY{o}{/}\PY{n}{combine}\PY{p}{[}\PY{l+s+s1}{\PYZsq{}}\PY{l+s+s1}{Clicks\PYZus{}con}\PY{l+s+s1}{\PYZsq{}}\PY{p}{]}
\PY{n}{combine}\PY{o}{.}\PY{n}{loc}\PY{p}{[}\PY{p}{:}\PY{p}{,}\PY{l+s+s1}{\PYZsq{}}\PY{l+s+s1}{n\PYZus{}exp}\PY{l+s+s1}{\PYZsq{}}\PY{p}{]} \PY{o}{=} \PY{n}{combine}\PY{p}{[}\PY{l+s+s1}{\PYZsq{}}\PY{l+s+s1}{Payments\PYZus{}exp}\PY{l+s+s1}{\PYZsq{}}\PY{p}{]}\PY{o}{/}\PY{n}{combine}\PY{p}{[}\PY{l+s+s1}{\PYZsq{}}\PY{l+s+s1}{Clicks\PYZus{}exp}\PY{l+s+s1}{\PYZsq{}}\PY{p}{]}
\PY{n}{combine}\PY{o}{.}\PY{n}{loc}\PY{p}{[}\PY{p}{:}\PY{p}{,}\PY{l+s+s1}{\PYZsq{}}\PY{l+s+s1}{n\PYZus{}sign}\PY{l+s+s1}{\PYZsq{}}\PY{p}{]} \PY{o}{=} \PY{n}{np}\PY{o}{.}\PY{n}{where}\PY{p}{(}\PY{n}{combine}\PY{p}{[}\PY{l+s+s1}{\PYZsq{}}\PY{l+s+s1}{n\PYZus{}exp}\PY{l+s+s1}{\PYZsq{}}\PY{p}{]}\PY{o}{\PYZgt{}}\PY{n}{combine}\PY{p}{[}\PY{l+s+s1}{\PYZsq{}}\PY{l+s+s1}{n\PYZus{}con}\PY{l+s+s1}{\PYZsq{}}\PY{p}{]}\PY{p}{,}\PY{l+m+mi}{1}\PY{p}{,}\PY{l+m+mi}{0}\PY{p}{)}
\PY{n}{combine}\PY{p}{[}\PY{p}{[}\PY{l+s+s1}{\PYZsq{}}\PY{l+s+s1}{g\PYZus{}sign}\PY{l+s+s1}{\PYZsq{}}\PY{p}{,}\PY{l+s+s1}{\PYZsq{}}\PY{l+s+s1}{n\PYZus{}sign}\PY{l+s+s1}{\PYZsq{}}\PY{p}{]}\PY{p}{]}
\end{Verbatim}
\end{tcolorbox}

            \begin{tcolorbox}[breakable, size=fbox, boxrule=.5pt, pad at break*=1mm, opacityfill=0]
\prompt{Out}{outcolor}{35}{\boxspacing}
\begin{Verbatim}[commandchars=\\\{\}]
    g\_sign  n\_sign
0        0       0
1        0       1
2        0       0
3        0       0
4        0       1
5        0       0
6        0       0
7        0       0
8        0       1
9        0       1
10       0       0
11       0       0
12       0       1
13       0       0
14       0       0
15       0       0
16       0       0
17       1       1
18       1       1
19       1       0
20       1       1
21       0       1
22       0       1
\end{Verbatim}
\end{tcolorbox}
        
    \begin{tcolorbox}[breakable, size=fbox, boxrule=1pt, pad at break*=1mm,colback=cellbackground, colframe=cellborder]
\prompt{In}{incolor}{21}{\boxspacing}
\begin{Verbatim}[commandchars=\\\{\}]
\PY{n}{g\PYZus{}sum} \PY{o}{=} \PY{n+nb}{sum}\PY{p}{(}\PY{n}{combine}\PY{p}{[}\PY{l+s+s1}{\PYZsq{}}\PY{l+s+s1}{g\PYZus{}sign}\PY{l+s+s1}{\PYZsq{}}\PY{p}{]}\PY{p}{)}
\PY{n}{n\PYZus{}sum} \PY{o}{=} \PY{n+nb}{sum}\PY{p}{(}\PY{n}{combine}\PY{p}{[}\PY{l+s+s1}{\PYZsq{}}\PY{l+s+s1}{n\PYZus{}sign}\PY{l+s+s1}{\PYZsq{}}\PY{p}{]}\PY{p}{)}
\PY{n}{total} \PY{o}{=} \PY{n}{combine}\PY{o}{.}\PY{n}{shape}\PY{p}{[}\PY{l+m+mi}{0}\PY{p}{]}
\PY{n+nb}{print}\PY{p}{(}\PY{l+s+s1}{\PYZsq{}}\PY{l+s+s1}{Number of Days where gross conversion in the experiment is larger than that in the control group:}\PY{l+s+s1}{\PYZsq{}}\PY{p}{,}\PY{n}{g\PYZus{}sum}\PY{p}{)}
\PY{n+nb}{print}\PY{p}{(}\PY{l+s+s1}{\PYZsq{}}\PY{l+s+s1}{Number of Days where net conversion in the experiment is larger than that in the control group:}\PY{l+s+s1}{\PYZsq{}}\PY{p}{,}\PY{n}{n\PYZus{}sum}\PY{p}{)}
\PY{n+nb}{print}\PY{p}{(}\PY{l+s+s1}{\PYZsq{}}\PY{l+s+s1}{Total number of days:}\PY{l+s+s1}{\PYZsq{}}\PY{p}{,}\PY{n}{total}\PY{p}{)}
\end{Verbatim}
\end{tcolorbox}

    \begin{Verbatim}[commandchars=\\\{\}]
Number of Days where gross conversion in the experiment is larger than that in
the control group: 4
Number of Days where net conversion in the experiment is larger than that in the
control group: 10
Total number of days: 23
    \end{Verbatim}

    Now let's calculate the p-value using the binomial distribution
equation.

 \(p(successes )=\frac{n!}{x!(n-x)!}p^x(1-p)^{n-x}\) 

    After we count the number of days in which the experiment has a higher
metric value than that of the control group (we define this event is a
``success''), we need to decide if the p-value is statistically
significant (\textless{} 0.05). On any day, if the event is a
``success'' or not is random, therefore \emph{p} = 0.5. n is total
number of days, which is 23. x is the number of days being a ``success''
(the value in the experiment group is larger than that in the control
group).

\(p-value\) is the probability of observing an event equal to or more
extreme than that observed. If we observed 4 successes, the \(p-value\)
for the test is:

\(P(x<=4)=P(0)+P(1)+P(2)+P(3)+P(4)\).

Because this is a two-tailed test, \(p-value\) will be doubled and
compared to 0.05.

    \begin{tcolorbox}[breakable, size=fbox, boxrule=1pt, pad at break*=1mm,colback=cellbackground, colframe=cellborder]
\prompt{In}{incolor}{22}{\boxspacing}
\begin{Verbatim}[commandchars=\\\{\}]
\PY{k+kn}{from} \PY{n+nn}{scipy}\PY{n+nn}{.}\PY{n+nn}{stats} \PY{k+kn}{import} \PY{n}{binom}
\PY{n}{n} \PY{o}{=} \PY{l+m+mi}{23}
\PY{n}{p} \PY{o}{=} \PY{l+m+mf}{0.5}
\PY{c+c1}{\PYZsh{} define a function that calculate the two\PYZhy{}tailed p\PYZhy{}value for a given x}
\PY{k}{def} \PY{n+nf}{twotail\PYZus{}pvalue}\PY{p}{(}\PY{n}{x}\PY{p}{,}\PY{n}{n}\PY{p}{,}\PY{n}{p}\PY{p}{)}\PY{p}{:}
    \PY{n}{p\PYZus{}x} \PY{o}{=} \PY{n+nb}{sum}\PY{p}{(}\PY{n}{binom}\PY{o}{.}\PY{n}{pmf}\PY{p}{(}\PY{n}{i}\PY{p}{,}\PY{n}{n}\PY{p}{,}\PY{n}{p}\PY{p}{)} \PY{k}{for} \PY{n}{i} \PY{o+ow}{in} \PY{n+nb}{range}\PY{p}{(}\PY{n}{x}\PY{o}{+}\PY{l+m+mi}{1}\PY{p}{)}\PY{p}{)}
    \PY{k}{return} \PY{n+nb}{round}\PY{p}{(}\PY{l+m+mi}{2}\PY{o}{*}\PY{n}{p\PYZus{}x}\PY{p}{,}\PY{l+m+mi}{4}\PY{p}{)}
\end{Verbatim}
\end{tcolorbox}

    \begin{tcolorbox}[breakable, size=fbox, boxrule=1pt, pad at break*=1mm,colback=cellbackground, colframe=cellborder]
\prompt{In}{incolor}{23}{\boxspacing}
\begin{Verbatim}[commandchars=\\\{\}]
\PY{n}{g\PYZus{}pvalue} \PY{o}{=} \PY{n}{twotail\PYZus{}pvalue}\PY{p}{(}\PY{n}{g\PYZus{}sum}\PY{p}{,}\PY{n}{n}\PY{p}{,}\PY{n}{p}\PY{p}{)}
\PY{n}{n\PYZus{}pvalue} \PY{o}{=} \PY{n}{twotail\PYZus{}pvalue}\PY{p}{(}\PY{n}{n\PYZus{}sum}\PY{p}{,}\PY{n}{n}\PY{p}{,}\PY{n}{p}\PY{p}{)}
\PY{n+nb}{print}\PY{p}{(}\PY{n}{g\PYZus{}pvalue}\PY{p}{)}
\PY{n+nb}{print}\PY{p}{(}\PY{n}{n\PYZus{}pvalue}\PY{p}{)}
\end{Verbatim}
\end{tcolorbox}

    \begin{Verbatim}[commandchars=\\\{\}]
0.0026
0.6776
    \end{Verbatim}

    \begin{longtable}[]{@{}ccc@{}}
\toprule
\begin{minipage}[b]{0.21\columnwidth}\centering
Evaluation metrics\strut
\end{minipage} & \begin{minipage}[b]{0.29\columnwidth}\centering
p-value for the sign test\strut
\end{minipage} & \begin{minipage}[b]{0.42\columnwidth}\centering
Statiscally significant at alpha=0.05\strut
\end{minipage}\tabularnewline
\midrule
\endhead
\begin{minipage}[t]{0.21\columnwidth}\centering
Gross conversion\strut
\end{minipage} & \begin{minipage}[t]{0.29\columnwidth}\centering
0.0026\strut
\end{minipage} & \begin{minipage}[t]{0.42\columnwidth}\centering
Yes\strut
\end{minipage}\tabularnewline
\begin{minipage}[t]{0.21\columnwidth}\centering
Net conversion\strut
\end{minipage} & \begin{minipage}[t]{0.29\columnwidth}\centering
0.6776\strut
\end{minipage} & \begin{minipage}[t]{0.42\columnwidth}\centering
No\strut
\end{minipage}\tabularnewline
\bottomrule
\end{longtable}

    The sensitivity of a sign test is usually lower than that of a effect
size test. Still, we get the same conclusion from the sign test as we
get from the effect size test above. The change in gross conversion is
significant, while the change in net conversion is not.

    \hypertarget{conclusion-and-recommendation}{%
\subsection{7. Conclusion and
Recommendation}\label{conclusion-and-recommendation}}

    We wanted to determine if making a change after a student clicks the
``start free trial'' button can have a significant impact on gross
conversion rate and net conversion. Therefor, we designed an A/B Testing
experiment. Udacity students (tracked by cookies) were directed randomly
into two groups, control and experiment. After clicking the ``start free
trial'' button, the experiment group was asked how many hours per week
they can devote to learning, while the control group was not.

Three invariant metrics (number of cookies, number of clicks and
Click-Through-Probability) were chosen and have passed sanity checks.
Gross conversion (enrollments/cookies) and net conversion
(payments/cookies) were chosen as evaluation metrics. Practical
significance threasholds (Dmin) were set for each evaluation metrics.

The null hypothesis H0 that there is no significant difference in the
evaluation metrics between the two groups. To reject the null
hypothesis, the difference between the groups should be statitically
significant, as well as exceed Dmin, for \textbf{all} evaluation
metrics.

The experiment results showed that gross conversion was found to be
statisticcaly significant with alpha = 0.05, and exceeded the practical
significance threahold. Net conversion, on the other hand, is neither
statistically significant nor practically significant.

    The purpose of the A/B testin experiment was to determine if by adding a
step qualifying if a student can dedicate enough study time, we can
improve the overall student experimence, thus reducing the number of
frustrated students who left the free trial because they didn't have
enough time; while at the same time without significantly reducing the
number of students to continue past the free trial. A statistically and
practically significant drop in gross conversion was observed; however,
no significant change in net conversion was observed. This means a
descrease in enrollment, and at the same time no increase in students
staying past the 14 day free trial leading to payment. Based on the
results, we do not recommend lauching the change at the moment without
conducting follow-up experiments.

    \hypertarget{follow-up-experiment}{%
\subsection{8. Follow-up Experiment}\label{follow-up-experiment}}

    Retention (payments/enrollments) was initially chosen as an evaluation
metrics but we didn't run the experiment using this metric or calculate
its statistical significance, because it would take about 5 months to
achieve the required pageviews. In reality where data are different and
it might not need 5 months to achieve the required sample size, a
company might decide retention is an important metric we are interested
in and we want to keep track of it. If a statistically significant and
practically significant retention change is observed, it means we are
seeing more students are staying and paying after the 14 day trial
period ends, out of all the students who are enrolled, and we would
recommend launching the experiment.


    % Add a bibliography block to the postdoc
    
    
    
\end{document}
